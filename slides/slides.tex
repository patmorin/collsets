\documentclass[xcolor=dvipsnames]{beamer}
\usepackage{ods}
%\usepackage{ods-figs}
\usepackage[cm]{sfmath}
\usepackage[utf8]{inputenc}
%\usepackage{enumitem}
%\usepackage{enumitem}
%\setitemize{itemsep=1.5ex}
%\setlength{\leftmargini}{0pt}
\usepackage{array}

\newcommand{\R}{\mathbb{R}}
\newcommand{\dual}[1]{#1^\star}
\newcommand{\Fary}{F\'ary}

\title{Dual Circumference and Free Sets}
\author{Vida Dujmović \and Pat Morin}
\titlegraphic{\includegraphics[height=1em]{by}}

\begin{document}

\begin{frame}
  \titlepage
\end{frame}

\begin{frame}
   \frametitle{The Setup}

   \begin{itemize}
      \item $G$ is an $n$-vertex plane graph.
   \end{itemize}
\end{frame}

\begin{frame}
  \frametitle{Proper Good Curves}
  
  \begin{itemize}
     \item A curve $C$ is 
     \begin{itemize}
        \item \emph{good} if it contains a point \emph{in} the outer face of $G$
        \item \emph{proper} if the intersection of $C$ with each edge of $G$ is
         \begin{itemize}
            \item empty; or
            \item a single point (maybe a vertex); or
            \item the entire edge.
         \end{itemize}
     \end{itemize}
  \end{itemize}
\end{frame}

\begin{frame}
  \frametitle{Free Sets}

   \begin{itemize}
     \item A set $F\subset V(G)$ is \emph{free} if for every
     $X\subset\R^2$ with $|X|=|F|$, $G$ has a \emph{non-crossing
     straight-line} drawing in which the vertices of $F$ are drawn on
     the points of $X$.

     \item Free sets have applications to
     \begin{itemize}
        \item untangling, 
        \item column planarity, 
        \item universal point subsets, 
        \item partial simultaneous geometric drawings.
        \item \ldots
     \end{itemize}
   \end{itemize}
\end{frame}

\begin{frame}
   \frametitle{The Free-Set Theorem}
   
   \begin{itemize}
       \item[]\textbf{Theorem (Dujmovi\'c--Frati--Gon\c{c}alves--M.--Rote 2019):} A set $F\subset V(G)$ is a \emph{free set} if and only if there is a proper good curve that contains $F$.
   \end{itemize}
\end{frame}

\begin{frame}
  \frametitle{Proper Good Curves and Dual Cycles} 

  \begin{itemize}[<+->]
     \item Every proper-good curve containing $k$ vertices gives a dual
       cycle of length at least $k$.
     \item What about the other direction?
  \end{itemize}
\end{frame}

\begin{frame}
   \frametitle{Pause}

   \begin{itemize}
      \item From this point on $G$ is (wlog) a triangulation.
      \item $G$ has maximum degree $\Delta$.
      \item $\dual{G}$ is the dual of $G$, a 3-connected cubic planar
      graph whose largest face has size at most $\Delta$
   \end{itemize}
\end{frame}



\begin{frame}
   \frametitle{Dual Cycles---Circumference}

   \begin{itemize}[<+->]
      \item The longest cycle in $\dual{G}$ is called its \emph{circumference} 
      \item Circumference of $n$-vertex 3-connected cubic (planar) $G$
      \begin{itemize}
        \item Tait's Conjecture (1884): $\forall \dual{G}: c(\dual{G}) = n$
        \item Disproved by Tutte (1946): $\exists \dual{G}: c(\dual{G})< n$
        \item Gr\"unbaum Walther (1973): $\exists \dual{G}: c(\dual{G}) = O(n^{0.9859})$.
        \item Barnette (XXXX): $\forall \dual{G}: c(\dual{G})=\Omega(\log n)$.
        \item Bondy and Simonovits (XXXX): $\forall \dual{G}: c(\dual{G})=e^{\Omega(\sqrt{\log n})}$
        \item Jackson (XXXX): $\forall \dual{G}: c(\dual{G}) = \Omega(n^{0.6942})$
        \item Billinski et al. (2011): $\forall \dual{G}: c(\dual{G}) = \Omega(n^{0.7532})$
        \item Liu, Yu, Zhang (2019): $\forall \dual{G}: c(\dual{G}) = \Omega(n^{0.8})$
      \end{itemize}
   \end{itemize}
\end{frame}

\begin{frame}
   \frametitle{Dual Cycles and Proper Good Curves}

   \begin{itemize}[<+->]
      \item Every dual cycle defines a proper good curve
      \item But the curve contains no vertices!
      \item Can we ``bend'' the curve to pick up some vertices?
      \item Sometimes yes, sometimes no
   \end{itemize}

\end{frame}

\begin{frame}
   \frametitle{Dual Cycles and Proper Good Curves}

   \begin{itemize}[<+->]
      \item A cycle $C$ in $\dual{G}$
      \begin{itemize}
         \item \emph{touches} $f$ if $f\cap C\neq\emptyset$
         \item \emph{caresses} $f$ if $f\cap C$ is a path
         \item \emph{pinches} $f$ if $f\cap C$ is a cycle or has more
               than one connected component.
      \end{itemize}
      \item $\tau$ouched, $\rho$inched, $\kappa$aressed
      \item $\tau = \rho + \kappa$
   \end{itemize}

   \begin{center}
      \includegraphics[height=.5\textheight]{figs/t0t1-2}
   \end{center}
\end{frame}

\begin{frame}
  \frametitle{Dual Cycles and Proper Good Curves}

  \begin{itemize}[<+->]
    \item[]\textbf{Theorem:} If $C$ caresses $\kappa$ faces of $\dual{G}$ then there is a proper good curve $\tilde{C}$ that contains at least $\kappa/4$ vertices of $G$.
    \item[]\textit{Proof:} Take an independent set $\dual{S}$ of caressed faces of $\dual{G}$ and ``bend'' $C$ so that it passes through each vertex of $S$.
  \end{itemize}
\end{frame}


\begin{frame}
  \frametitle{Summary So Far}

  
  \begin{center}
    \uncover<4->{{\color{blue}cycle in $G$ of length $\ell=c\kappa\Delta^4$ \\ $\Downarrow$} \\}
    \uncover<1->{cycle in $\dual{G}$ that caresses $\kappa$ faces}\\
    \uncover<2->{$\Downarrow$ \\
    proper good curve that contains $\kappa/4$ vertices of $G$}\\
    \uncover<3->{$\Downarrow$ \\
    free set of size $\kappa/4$}
  \end{center}
\end{frame}

\begin{frame}
  \frametitle{The Main Theorem}

  \begin{itemize}
     \item[]\textbf{Theorem:} If $G$ has maximum degree $\Delta$ and $\dual{G}$ has a cycle of length $\ell$, then $\dual{G}$ has a cycle $C'$ that caresses $\kappa=\Omega(\ell/\Delta^4)$ faces.
  \end{itemize}
\end{frame}

\begin{frame}
  \frametitle{Proof Ideas}

  
  \begin{itemize}
     \item Want to show that $C$ $\kappa$aresses many faces
     \item $C$ has length $\ell$
     \item each face of $\dual{G}$ has most $\Delta$ edges
     \item[$\therefore$] $C$ $\tau$ouches at least $\ell/\Delta$ faces
     \item Recall $\tau = \kappa + \rho$
     \item If $\kappa =\Omega(\ell/\Delta)$ we are done
     \item Otherwise $C$ $\rho$inches at least $\rho = \Omega(\ell/\Delta)$ faces
  \end{itemize}
\end{frame}


\begin{frame}
  \frametitle{A Bad Example}

  \begin{center}
     \includegraphics[width=.9\textwidth]{figs/two-caressed}
  \end{center}
  \begin{itemize}
    \item $\tau = \kappa + \rho$
    \item $\tau = 2k+3$, $\kappa = 4$, $\rho = 2k-1$
  \end{itemize}
\end{frame}


\begin{frame}
  \frametitle{Proof Idea}

  \begin{center}
     \includegraphics[height=.2\textheight]{figs/two-caressed}
  \end{center}
  \begin{itemize}
    \item Define two trees $T_0$ (inside $C$) and $T_1$ (outside $C$)
    \item Show that:
    \begin{itemize}
      \item Each leaf of $T_i$ contains a caressed face
      \item Each $T_i$ has $\Omega(\ell/\Delta)$ nodes
    \end{itemize}
    \item We are done if either $T_i$ has many leaves, so each $T_i$ must have few leaves
    \item[$\therefore$] each $T_i$ has many degree 2 nodes
    \item With enough degree-2 nodes, we can perform a \emph{surgery} that increases the number of leaves in $T_0$.
  \end{itemize}
\end{frame}


\begin{frame}
  \frametitle{The Trees $T_0$ and $T_1$}

  
\end{frame}


%      \item For a node $u$ of $T_i$,
%        \[ \rho_u \le 2(\kappa_u + \delta_u) \Leftrightarrow 
%           \tau_u \le 3\kappa_u + 2\delta_u \]
%    \end{itemize}
%    \item So if $\tau_u$ is big then either
%    \begin{itemize}
%       \item $\kappa_u$ is big; or
%       \item $\delta_u$ is big
%    \end{itemize}
%  \end{itemize}
%\end{frame}
%






%\closing

\end{document}

