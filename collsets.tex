\documentclass{patmorin}
%\usepackage[top=0.85in, bottom=0.85in, left=0.85in, right=0.85in]{geometry}

%%%%%%%%%%%%%%%%%%%%%%%%%%%%%%%%%%%%%%%%%%%%%%%%%%%%%%%%%%%%%%%%%%%%%%%%%%%%%%%%%%%%%%%%%%
%% Packages
%%%%%%%%%%%%%%%%%%%%%%%%%%%%%%%%%%%%%%%%%%%%%%%%%%%%%%%%%%%%%%%%%%%%%%%%%%%%%%%%%%%%%%%%%%
\usepackage{amsmath}
\usepackage{amsfonts}
\usepackage{amsthm}
\usepackage{graphicx}
%\usepackage{xspace}
%\usepackage{wrapfig}
\newenvironment{wrapfigure}[9]{\begin{figure}}{\end{figure}}
\newcommand{\Vspace}[1]{}
\usepackage{enumerate}
\usepackage{cite}
\usepackage{pat}
\usepackage{paralist}
\usepackage{hyperref}
\hypersetup{colorlinks=true, linkcolor=linkblue,  anchorcolor=linkblue,
citecolor=linkblue, filecolor=linkblue, menucolor=linkblue,
urlcolor=linkblue, pdfcreator=Me, pdfproducer=Me} 
\setlength{\parskip}{1ex}
\usepackage{algorithm}
\usepackage{subfig}
%\usepackage{subfigure}
\usepackage{array}
\usepackage[noend]{algpseudocode}
\usepackage[usenames]{xcolor}
\usepackage{stmaryrd} % for lightning-symbol
\usepackage{mathtools} % for mathclap in \conf-definition
\usepackage{todonotes}
%\usepackage{compress}
%\usepackage{times}
\usepackage{lineno}

\title{\MakeUppercase{Dual Cycles and Collinear Sets in Triangulations}%
    \thanks{This work was partly funded by NSERC and MRI.}}

\author{Vida Dujmovi\'c\thanks{Department of Computer Science and Electrical Engineering, University of Ottawa}\,\, and 
        Pat Morin\thanks{School of Computer Science, Carleton University}}

\newcommand{\dual}[1]{{#1}^\star}
\newcommand{\note}[2]{{\color{red}#1:~#2}}

\begin{document}
\maketitle


\begin{abstract}
   We show that, if a $n$-vertex triangulation $T$ of maximum degree
   $\Delta$ has a dual that contains a cycle of length $\ell$, then
   $T$ has a plane straight-line drawing in which some \emph{collinear
   set} of $\Omega(\ell/\Delta^2)$ vertices lie on a line.  Using the
   current lower bounds on the length of longest cycles in 3-regular
   3-connected graphs, this implies that $T$ has a collinear set of
   size $\Omega(n^{0.8}/\Delta^2)$.  Such collinear sets have numerous
   results in graph drawing and related areas.
\end{abstract}

\section{Introduction}

For a planar graph $G$, we say that a set $S\subseteq V(G)$ is
a \emph{collinear set} if $G$ has a non-crossing straight-line drawing in
which the vertices of $S$ are all collinear.  The \emph{dual} $\dual{G}$
of a plane graph $G$ is the graph whose vertex set $V(\dual{G})$ is
the set of faces in $G$ and in which $f,g\in E(\dual{G})$ if and only
if the face $f$ and the face $g$ have at least one edge in common.
The \emph{circumference}, $c(G)$ of a graph $G$ is the length of its
longest cycle. We prove the following theorem:

\begin{thm}\thmlabel{main}
  Let $T$ be a triangulation of maximum degree $\Delta$ whose dual
  $\dual{T}$ has circumference $\ell$. Then $T$ has a collinear set of
  size $\Omega(\ell/\Delta^2)$.
\end{thm}

The dual of a triangulation is a 3-connected cubic planar graph.
The study of the circumference of 3-connected cubic planar graphs
has a long and rich history going back at least 1884 when Tait
\cite{tait:listings},conjectured that every such graph is Hamiltonian.  In
1946, Tait's conjecture was disproved by Tutte who gave a non-Hamiltonian
46-vertex example \cite{tutte:on}.  Repeatedly replacing vertices of
this graph with copies of itself gives a family of graphs, $\langle G_i:i\in
\Z\rangle$ in which $G_i$ has $46\cdot 45^i$ vertices and circumference at
most $45\cdot44^i$.  Stated another way, $n$-vertex members of the
family have circumference $O(n^a)$, for $a=\log_{44}(45)) < n^{0.9941}$.
The current best upper bound of this type is due to Gr\"unbaum and
Walther \cite{grunbaum.walther:shortness} who construct a 24-vertex
non-Hamiltonian cubic 3-connected planar graph, resulting in a family
of graphs in which $n$-vertex members have circumference $O(n^{\alpha})$
for $\alpha=\log_{23}(22)< 0.9859$.

A series of results has steadily improved the lower bounds on the
circumference of $n$-vertex  (not necessarily planar) 3-connected cubic
graphs.  Barnette \cite{barnette:4} showed that, for every $n$-vertex
3-connected cubic graph $G$, $c(G)=\Omega(\log n)$.  Bondy and Simonovits
\cite{bondy.simonovits:7} improved this bound to $e^{\Omega(\sqrt{\log
n})}$ and conjectured that it can be improved to $\Omega(n^c)$ for $c>0$.
Jackson \cite{jackson:8} confirmed this bound with $c=\log_2(1+\sqrt{5})-1
> 0.6942$.  Billinksi \etal\ \cite{billinksi.jacdson.ea:6} improved this
to the root of $4^{1/c}-3^{1/c}=2$, which is $c>0.7532$.  The current
record is held by Liu, Yu, and Zhang \cite{liu.yu.zhang:circumference}
who show that $c>0.8$.  Together with \thmref{main}, this result implies
the following corollary:

\begin{cor}\corlabel{main}
  Every $n$-vertex triangulation of maximum degree $\Delta$ contains a
  collinear set of size $\Omega(n^{0.8}/\Delta^2)$.
\end{cor}

A longstanding open problem on circumference is \emph{Barnette's
Conjecture}, which asserts that every \emph{bipartite} 3-connected cubic
planar graph is Hamiltonian. Note that, if Barnette's conjecture is true,
this would imply that every triangulation in which every vertex has even
degree contains a collinear set of linear size.  \note{PM}{Think about
whether this even makes sense. For example, the assumption can only be
true when $n$ is odd.  Maybe there's a little trick we can do to make
it more generally applicable, like when only a small number, $k$, of
vertices have odd degree.}



\section{Proof of \thmref{main}}

Let $G$ be a plane graph. A \emph{proper good curve} $C$ for $G$ is a
Jordan curve with the following properties:
\begin{enumerate}
  \item proper: for any edge $xy$ of $G$, $C$ either contains $xy$, intersects
  $xy$ in a single point (possibly an endpoint) or $C$ is disjoint
  from $xy$; and
  \item good: $C$ contains at least one point in the interior of
  the outer face of $T$.
\end{enumerate}

Da Lozzo \etal\ show that proper good curves define collinear sets:

\begin{thm}
  In a plane graph $G$, a set $S\subseteq V(G)$ is a collinear set if
  and only if there is a proper curve for $G$ that contains $S$.
\end{thm}

For a triangulation $T$, let $v(T)$ denote the size of the largest
collinear set in $T$.  We will show that, for any triangulation $T$
of maximum degree $\Delta$
whose dual is $T^*$, $v(T)=\Theta(c(\dual{T})/\Delta^2)$ by demonstrating a
relationship between proper good curves in $T$ and cycles in $\dual{T}$.

In one direction, this result is easy, as used by Ravsky and Verbitsky
\cite{ravsky.verbitsky:on,ravsky.verbitsky:on-arxiv}.  If $T$ is a
triangulation that has a proper good curve $C$ containing $k$ vertices,
then a slight deformation of $C$ produces a proper-good curve that
contains no vertices. This curve intersects a cyclic sequence of faces
$f_0,\ldots,f_{k'-1}$ of $T$ with $k'\ge k$.  In this sequence, $f_i$ and
$f_{(i+1)\bmod k'}$ share an edge, for every $i\in\{0,\ldots,k'-1\}$, so
this sequence is a closed walk in the dual $\dual{T}$ of $T$.  Property~1
of good curves and the fact that each face of $T$ is a triangle ensures
that $f_i\neq f_j$ for any $i\neq j$, so this sequence is a cycle in
the dual of length $k'\ge k$.  Therefore, $c(\dual{T})\ge v(T)$.

\subsection{Faces that are Touched, Pinched, and Good}

Let $C$ be a cycle in $\dual{T}$.  We say that $C$ is non-trivial if it
is not a single face of $\dual(T)$.   We say that a face $f$ of $\dual{T}$ 
\begin{enumerate}
  \item is \emph{touched} by $C$ if $f\cap C\neq \emptyset$;
  \item is \emph{pinched} by $C$ if $f\cap C$ has more than one connected component;
  \item is \emph{good} for $C$ if it is touched but not pinched by $C$.
\end{enumerate}

\begin{lem}
   Let $T$ be a triangulation and $C$ be a non-trivial cycle in $\dual{T}$ for which $k$ vertices of $\dual{T}$ are good.  Then $T$ has a proper-good curve that contains at least $k/4$ vertices, i.e., $v(T)\ge k/4$.
\end{lem}

\begin{proof}
  Let $F$ be the set of faces in $\dual{T}$ that are good for $C$. Each
  element $u\in F$ corresponds to a vertex of $T$ so we will treat $F$
  as a set of vertices in $T$.  Consider the subgraph $T[F]$ of $T$
  induced by $F$.  This graph is planar and has $k$ vertices. Therefore,
  by the 4-Colour Theorem it contains an independent set $F'\subset F$
  of size at least $k/4$.

  We claim that $T$ has a proper-good curve that contains all the vertices
  in $F'$.  To see this, first observe that the cycle $C$ in $\dual{T}$
  already defines a proper-good curve (that does not contain any vertices
  of $T$) that we will also call $C$.  We will perform surgery on $C$
  so that it contains all the elements of $F'$.

  For each vertex $u\in F'$, let $w_0,\ldots,w_{d-1}$ denote the
  neighbours of $u$ in cyclic order.  The curve $C$ intersects some
  contiguous subsequence $uw_i,\ldots,uw_j$ of the edges adjacent
  to $u$.  Since $C$ is non-trivial, this sequence does not contain all
  edges incident to $u$. In particular, the curve $C$ crosses the edge
  $w_{i-1}w_i$, then crosses
  $uw_i,\ldots,uw_j$, and then crosses the edge $w_j w_{j+1}$.  We modify
  $C$ by removing the portion between the first and last of these crossings
  and replacing it with a curve that contains $u$ and is contained in the
  two triangles $w_{i-1}uw_i$ and $w_{j-1}uw_j$.

  After performing this surgery for each $u\in F'$ we have a curve $C'$
  that contains every vertex $u\in F'$.  All that remains is verify that
  $C'$ is good and proper for $T$. That $C'$ is good for $T$ is
  obvious.  That $C'$ is proper for $T$ follows the following two observations:
  (i)~$C'$ does not contain any two adjacent vertices (since $F'$ is an
  independent set); and (ii)~if $C'$ contains a vertex $u$, then it does
  not intersect the interior of any edge incident to $u$.
\end{proof}



To prove that $v(T) = \Omega(c(\dual{T}))$ we begin with a 


\section{}

\newpage
\bibliographystyle{plain}
\bibliography{freecoll}

\end{document}









