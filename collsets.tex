\documentclass{patmorin}
%\usepackage[top=0.85in, bottom=0.85in, left=0.85in, right=0.85in]{geometry}

%%%%%%%%%%%%%%%%%%%%%%%%%%%%%%%%%%%%%%%%%%%%%%%%%%%%%%%%%%%%%%%%%%%%%%%%%%%%%%%%%%%%%%%%%%
%% Packages
%%%%%%%%%%%%%%%%%%%%%%%%%%%%%%%%%%%%%%%%%%%%%%%%%%%%%%%%%%%%%%%%%%%%%%%%%%%%%%%%%%%%%%%%%%
\usepackage{amsmath}
\usepackage{amsfonts}
\usepackage{amsthm}
\usepackage{graphicx}
%\usepackage{xspace}
%\usepackage{wrapfig}
\newcommand{\Vspace}[1]{}
\usepackage{enumerate}
\usepackage{cite}
\usepackage{pat}
\usepackage{paralist}
\usepackage{hyperref}
\hypersetup{colorlinks=true, linkcolor=linkblue,  anchorcolor=linkblue,
citecolor=linkblue, filecolor=linkblue, menucolor=linkblue,
urlcolor=linkblue, pdfcreator=Me, pdfproducer=Me} 
\setlength{\parskip}{1ex}
\usepackage{algorithm}
\usepackage{subfig}
\usepackage{array}
\usepackage[noend]{algpseudocode}
\usepackage[usenames]{xcolor}
\usepackage{stmaryrd} % for lightning-symbol
\usepackage{mathtools} % for mathclap in \conf-definition
\usepackage{todonotes}
%\usepackage{compress}
%\usepackage{times}
\usepackage{lineno}

\title{\MakeUppercase{Dual Circumference and Collinear Sets}%
    \thanks{This work was partly funded by NSERC and MRI.}}

\author{Vida Dujmovi\'c\thanks{Department of Computer Science and Electrical Engineering, University of Ottawa}\,\, and 
        Pat Morin\thanks{School of Computer Science, Carleton University}}

\newcommand{\dual}[1]{{#1}^\star}
\newcommand{\note}[2]{{\color{red}#1:~#2}}

\begin{document}
\maketitle


\begin{abstract}
   We show that, if a $n$-vertex triangulation $T$ of maximum degree
   $\Delta$ has a dual that contains a cycle of length $\ell$, then
   $T$ has a plane straight-line drawing in which some \emph{collinear
   set} of $\Omega(\ell/\Delta^2)$ vertices lie on a line.  Using the
   current lower bounds on the length of longest cycles in 3-regular
   3-connected graphs, this implies that $T$ has a collinear set of
   size $\Omega(n^{0.8}/\Delta^2)$.  Such collinear sets have numerous
   results in graph drawing and related areas.
\end{abstract}

\section{Introduction}

For a planar graph $G$, we say that a set $S\subseteq V(G)$ is
a \emph{collinear set} if $G$ has a non-crossing straight-line drawing in
which the vertices of $S$ are all collinear.  The \emph{dual} $\dual{G}$
of a plane graph $G$ is the graph whose vertex set $V(\dual{G})$ is
the set of faces in $G$ and in which $f,g\in E(\dual{G})$ if and only
if the face $f$ and the face $g$ have at least one edge in common.
The \emph{circumference}, $c(G)$ of a graph $G$ is the length of its
longest cycle. We prove the following theorem:

\begin{thm}\thmlabel{main}
  Let $T$ be a triangulation of maximum degree $\Delta$ whose dual
  $\dual{T}$ has circumference $\ell$. Then $T$ has a collinear set of
  size $\Omega(\ell/\Delta^2)$.
\end{thm}

The dual of a triangulation is a 3-connected cubic planar graph.
The study of the circumference of 3-connected cubic planar graphs
has a long and rich history going back at least 1884 when Tait
\cite{tait:listings},conjectured that every such graph is Hamiltonian.  In
1946, Tait's conjecture was disproved by Tutte who gave a non-Hamiltonian
46-vertex example \cite{tutte:on}.  Repeatedly replacing vertices of
this graph with copies of itself gives a family of graphs, $\langle G_i:i\in
\Z\rangle$ in which $G_i$ has $46\cdot 45^i$ vertices and circumference at
most $45\cdot44^i$.  Stated another way, $n$-vertex members of the
family have circumference $O(n^a)$, for $a=\log_{44}(45)) < n^{0.9941}$.
The current best upper bound of this type is due to Gr\"unbaum and
Walther \cite{grunbaum.walther:shortness} who construct a 24-vertex
non-Hamiltonian cubic 3-connected planar graph, resulting in a family
of graphs in which $n$-vertex members have circumference $O(n^{\alpha})$
for $\alpha=\log_{23}(22)< 0.9859$.

A series of results has steadily improved the lower bounds on the
circumference of $n$-vertex  (not necessarily planar) 3-connected cubic
graphs.  Barnette \cite{barnette:4} showed that, for every $n$-vertex
3-connected cubic graph $G$, $c(G)=\Omega(\log n)$.  Bondy and Simonovits
\cite{bondy.simonovits:7} improved this bound to $e^{\Omega(\sqrt{\log
n})}$ and conjectured that it can be improved to $\Omega(n^\alpha)$ for $\alpha>0$.
Jackson \cite{jackson:8} confirmed this bound with $\alpha=\log_2(1+\sqrt{5})-1
> 0.6942$.  Billinksi \etal\ \cite{billinksi.jacdson.ea:6} improved this
to the root of $4^{1/\alpha}-3^{1/\alpha}=2$, which implies $\alpha>0.7532$.  The current
record is held by Liu, Yu, and Zhang \cite{liu.yu.zhang:circumference}
who show that $\alpha>0.8$.  Together with \thmref{main}, this result implies
the following corollary:

\begin{cor}\corlabel{main}
  Every $n$-vertex triangulation of maximum degree $\Delta$ contains a
  collinear set of size $\Omega(n^{0.8}/\Delta^2)$.
\end{cor}

A longstanding open problem on circumference is \emph{Barnette's
Conjecture}, which asserts that every \emph{bipartite} 3-connected cubic
planar graph is Hamiltonian. Note that, if Barnette's conjecture is true,
this would imply that every triangulation in which every vertex has even
degree contains a collinear set of linear size.  \note{PM}{Think about
whether this even makes sense. For example, the assumption can only be
true when $n$ is odd.  Maybe there's a little trick we can do to make
it more generally applicable, like when only a small number, $k$, of
vertices have odd degree.}



\section{Proof of \thmref{main}}

Let $G$ be a plane graph.  We treat the vertices of $G$ as points,
the edges of $G$ as closed curves and the faces of $G$ as closed sets
(so that a face contains all the edges on its boundary and an edge
contains both its endpoints).  Whenever we consider subgraphs of $G$
we treat them as having the same embedding as $G$.  Similarly, if we
consider a graph $\bar{G}$ that is homeomorphic to $G$ then we assume
that the edges of $\bar{G}$---each of which repreesents a path in $G$
whose internal vertices all have degree 2---inherit their embedding from
the paths they represent in $G$.

Finally, if we consider the dual $G^*$ of $G$ then we treat it as a
plane graph in which each vertex $f$ is represented as a point in the
interior of the face $f$ of $G$ that it represents.  The edges of $G^*$
are embedded so that an edge $fg$ is contained in the union of the two
faces $f$ and $g$ of $G$, it intersects the interior of exactly one
edge of $G$ that is common to $f$ and $G$, and this intersection
consists of a single point.

A \emph{proper good curve} $C$ for a plane graph $G$ is a
Jordan curve with the following properties:
\begin{enumerate}
  \item proper: for any edge $xy$ of $G$, $C$ either contains $xy$, intersects
  $xy$ in a single point (possibly an endpoint), or is disjoint
  from $xy$; and
  \item good: $C$ contains at least one point in the interior of
  the outer face of $G$.
\end{enumerate}

Da Lozzo \etal\ show that proper good curves define collinear sets:

\begin{thm}\thmlabel{dalozzo}
  In a plane graph $G$, a set $S\subseteq V(G)$ is a collinear set if
  and only if there is a proper good curve for $G$ that contains $S$.
\end{thm}

For a triangulation $T$, let $v(T)$ denote the size of the largest
collinear set in $T$.  We will show that, for any triangulation $T$
of maximum degree $\Delta$
whose dual is $T^*$, $v(T)=\Theta(c(\dual{T})/\Delta^2)$ by demonstrating a
relationship between proper good curves in $T$ and cycles in $\dual{T}$.

In one direction, this result is easy, as used by Ravsky and Verbitsky
\cite{ravsky.verbitsky:on,ravsky.verbitsky:on-arxiv}.  If $T$ is a
triangulation that has a proper good curve $C$ containing $k$ vertices,
then a slight deformation of $C$ produces a proper good curve that
contains no vertices. This curve intersects a cyclic sequence of faces
$f_0,\ldots,f_{k'-1}$ of $T$ with $k'\ge k$.  In this sequence, $f_i$ and
$f_{(i+1)\bmod k'}$ share an edge, for every $i\in\{0,\ldots,k'-1\}$, so
this sequence is a closed walk in the dual $\dual{T}$ of $T$.  Property~1
of good curves and the fact that each face of $T$ is a triangle ensures
that $f_i\neq f_j$ for any $i\neq j$, so this sequence is a cycle in
the dual of length $k'\ge k$.  Therefore, $c(\dual{T})\ge v(T)$.

\subsection{Faces that are Touched, Pinched, and Caressed}

We say that a cycle $C$ in a plane graph is non-trivial if it is not
contained in a single face of the graph.  Throughout the remainder of
this section, $T$ is a triangulation whose dual is $\dual{T}$ and $C$
is a non-trivial cycle in $\dual{T}$.  Refer to \figref{touched-pinched-caressed}.
We say that a face $f$ of $\dual{T}$ 
\begin{enumerate}
  \item is \emph{touched} by $C$ if $f\cap C\neq \emptyset$;
  \item is \emph{pinched} by $C$ if $f\cap C$ has more than one connected component;
  \item is \emph{caressed} by $C$ if it is touched but not pinched by $C$.
\end{enumerate}

\begin{figure}
\begin{center}
	  \includegraphics{figs/tpc-2}
\end{center}
	  \caption{Faces of $T^*$ that are pinched and caressed by $C$. $C$ is bold, caressed faces are blue, pinched faces are pink, and untouched faces are unshaded.}
	  \figlabel{touched-pinched-caressed}
\end{figure}


Since $C$ is always the cycle of interest, we will say simply that a face
$f$ of $\dual{T}$ is touched, pinched, or caressed, without specifically
mentioning $C$.  We will frequently use the values $\tau$, $\rho$, and
$\kappa$ to denote the number of faces in some region that $\tau$ouched,
$\rho$inched or $\kappa$aressed.

\begin{lem}\lemlabel{cycle-to-curve}
   If $C$ caresses $\kappa$ faces of $\dual{T}$ then $T$ has a proper good
	curve that contains at least $\kappa/4$ vertices so, by \thmref{dalozzo}, $v(T)\ge \kappa/4$.
\end{lem}

\begin{proof}
  Let $F$ be the set of faces in $\dual{T}$ that are caressed by $C$. Each
  element $u\in F$ corresponds to a vertex of $T$ so we will treat $F$
  as a set of vertices in $T$.  Consider the subgraph $T[F]$ of $T$
  induced by $F$.  This graph is planar and has $k$ vertices. Therefore,
  by the 4-Colour Theorem it contains an independent set $F'\subset F$
  of size at least $\kappa/4$.

  We claim that $T$ has a proper good curve that contains all the vertices
  in $F'$.  To see this, first observe that the cycle $C$ in $\dual{T}$
  already defines a proper good curve (that does not contain any vertices
  of $T$) that we will also call $C$.  We will perform surgery on $C$
  so that it contains all the vertices in $F'$.

  For each vertex $u\in F'$, let $w_0,\ldots,w_{d-1}$ denote the
  neighbours of $u$ in cyclic order.  The curve $C$ intersects some
  contiguous subsequence $uw_i,\ldots,uw_j$ of the edges adjacent
  to $u$.  Since $C$ is non-trivial, this sequence does not contain all
  edges incident to $u$. In particular, the curve $C$ crosses the edge
  $w_{i-1}w_i$, then crosses
  $uw_i,\ldots,uw_j$, and then crosses the edge $w_j w_{j+1}$.  We modify
  $C$ by removing the portion between the first and last of these crossings
  and replacing it with a curve that contains $u$ and is contained in the
  two triangles $w_{i-1}uw_i$ and $w_{j-1}uw_j$. (See \figref{cycle-to-curve}.)

  \begin{figure}
     \begin{center}
	\begin{tabular}{cc}
		\includegraphics{figs/cycle-to-curve-1} &
		\includegraphics{figs/cycle-to-curve-2}
	\end{tabular}
     \end{center}
     \caption{Transforming the dual cycle $C$ into a proper good curve $C'$ containing $u$.}
	  \figlabel{cycle-to-curve}
  \end{figure}

  After performing this surgery for each $u\in F'$ we have a curve $C'$
  that contains every vertex $u\in F'$.  All that remains is verify that
  $C'$ is good and proper for $T$. That $C'$ is good for $T$ is
  obvious.  That $C'$ is proper for $T$ follows the following two observations:
  (i)~$C'$ does not contain any two adjacent vertices (since $F'$ is an
  independent set); and (ii)~if $C'$ contains a vertex $u$, then it does
  not intersect the interior of any edge incident to $u$.
\end{proof}

\lemref{cycle-to-curve} reduces our problem to finding a cycle in
$\dual{T}$ that caresses many faces.  We do this by showing that any
long cycle in $\dual{T}$ can be transformed into a cycle that caresses
many faces of $\dual{T}$.  

We begin with a simple but useful lemma.  A path $P$ in $\dual{T}$ is
a \emph{chord path} (for $C$) if both endpoints on $P$ are in $V(C)$
and none of $P$'s edges are in $E(C)$.

\begin{lem}\lemlabel{one-caressed}
   Let $P$ be a chord path for $C$ and let $L$ and $R$ be the two faces
   of $P\cup C$ that have $P$ on their boundary. Then each of $L$ and $R$
   contain at least one face of $\dual{T}$ that is caressed by $C$.
\end{lem}

\begin{proof}
   It suffices to consider $R$, since $L$ is symmetric.  The proof is by
   induction on the number, $t$, of faces of $\dual{T}$ contained in $R$.
   If $t=1$, then $R$ is a face of $\dual{T}$ and it is caressed by $C$.

   If $t>1$, then consider the face $f$ of $\dual{T}$ that is contained
	in $R$ and has the first edge of $P$ on its boundary.  Refer to \figref{one-caressed}. Since $t>1$,
   $X=R\setminus f$ is non-empty. The set $X$ may have several connected
   components $X_1,\ldots,X_k$, but each $X_i$ has a boundary that contains a chord path $P_i$ for $C$.
	We can therefore apply induction on $P_1$ (or any $P_i$).
  \begin{figure}
     \begin{center}
	\begin{tabular}{cc}
		\includegraphics{figs/one-caressed-1} &
		\includegraphics{figs/one-caressed-2}
	\end{tabular}
     \end{center}
	  \caption{The proof of \lemref{one-caressed}.}
	  \figlabel{one-caressed}
  \end{figure}
\end{proof}

Refer to \figref{auxilliary}. \note{PM}{Use the graph in \figref{touched-pinched-caressed instead}.}
Consider the auxilliary
graph $H$ with vertex set $V(H)\subseteq V(\dual{T})$ and whose edge set
consist of the edges of $C$ plus those edges of $\dual{T}$ that belong
to any face pinched by $C$. Let $v_0,\ldots,v_{r-1}$ be the cyclic sequence of vertices on some face $f$ of $\dual{T}$ that is pinched by $C$.  
We identify two kinds of vertices that are \emph{special} with respect to $f$:
\begin{enumerate}
  \item A vertex $v_i$ is special of \emph{Type~A} if $v_{i-1}v_i$ is an edge of $C$ and $v_iv_{i+1}$ is not an edge of $C$.
  \item A vertex $v_i$ is special of \emph{Type~B} if $v_{i-1}v_i$ is not an edge of $C$ and $v_iv_{i+1}$ is an edge of $C$.
  \item A vertex $v_i$ is special of \emph{Type~Y} if $v_i$ not incident to any edge of $C$ and $v_i$ has degree 3 in $H$.
\end{enumerate}

  \begin{figure}
     \begin{center}
		\includegraphics{figs/touched-pinched-caressed-1} 
		\includegraphics{figs/touched-pinched-caressed-2}
     \end{center}
	  \caption{The auxilliary graph $H$ and the trees $T_0$ and $T_1$.}
	  \figlabel{auxilliary}
  \end{figure}


We say that a path $v_i,\ldots,v_j$ is a \emph{keeper} with respect to
$f$ if $v_i$ is special of Type~A, $v_j$ is special of Type~B, and none
of $v_{i+1},\ldots,v_{j-1}$ are special.  We let $\tilde{H}$ denote the
subgraph of $H$ containing all the edges of $C$ and all the edges of
all paths that are special with respect to some face $f$ of $\dual{T}$.

It is worth remarking at this point that, by definition, every keeper
is contained in the boundary of at least one face $f$ of $\dual{T}$.
This property will be useful shortly.

Let $\bar{H}$ denote the graph that is homeormophic to $\tilde{H}$ but does not
contain any degree 2 vertices.  That is, $\bar{H}$ is the minor of $\tilde{H}$
obtained by repeatedly contracting an edge incident a degree-2 vertex.
The graph $\bar{H}$ naturally inherits an embedding from the embedding of $\tilde{H}$ (which inherits its embedding from an embedding $H$, which inherits an embedding from $\dual{T})$.  This embedding partitions the edges of $\bar{H}$ into three sets:
\begin{enumerate}
  \item The set $B$ of edges that are contained in (the embedding of) $C$;
  \item The set $E_0$ of edges whose interiors are contained in the interior of (the embedding of) $C$; and
  \item The set $E_1$ of edges whose interiors are contained in the exterior of (the embedding of) $C$.
\end{enumerate}

Observe that, for each $i\in\{0,1\}$, the graph $H_i$ whose edges are
exactly those in $B\cup E_i$ is outerplanar.  Let $T_i$ be the subgraph of
$\dual{H_i}$ whose edges are all those dual to the edges of $E_i$. From
the outerplanarity of $H_i$, it follows that $T_i$ is a tree. Note that
the nodes of $T_i$, which are faces of $\bar{H}$, are subsets of the
plane obtained by taking the closure of the union of faces in $\dual{T}$.
In the following, when we say that a node $u$ of $T_i$ contains a face
$f$ of $\dual{T}$ we mean that $f$ is one of the faces of $\dual{T}$
whose union makes up $u$.


The following lemma allows us to direct our effort towards proving that
one of $T_0$ or $T_1$ has many leaves.

\begin{lem}
   Each leaf $u$ of $T_i$ contains at least one face of $\dual{T}$
   that is caressed by $C$.
\end{lem}

\begin{proof}
   The edge of $T_i$ incident to $u$ corresponds to a chord path $P$. The
   graph $P\cup C$ has two faces with $P$ on its boundary, one of which
   is $u$.  The lemma now follows immediately from \lemref{one-caressed}.
\end{proof}


\begin{lem}\lemlabel{many-good}
   Let $u$ be a node of $T_i$ and let $\rho_u$, $\kappa_u$, and $\delta_u$ denote the number of pinched faces of $\dual{T}$ in $u$, the number of caressed faces of $\dual{T}$ in $u$, and the degree of $u$ in $T_i$, respectively.  Then $\rho_u \le 2(\kappa_u+\delta_u)$.
\end{lem}

\begin{proof}
   The proof is a discharging argument.  We assign each pinched face of
   $u$ a single unit of charge, so that the total charge is $\rho_u$.
   We then describe a set of discharging rules that preserve the total
   charge.  After the application of these rules, pinched faces in $u$
   have no charge, each caressed face in $u$ has charge at most 2,
   and each keeper path in $u$ has charge at most 2.  Since there is a
   bijection between keeper paths in $u$ and edges of $T_i$ incident to
   $u$, this proves the result.

   We now describe the discharging rules, which are recursive and take
   as input a chord path $P$ that partitions $u$ into two parts $L$
   and $R$.  We require as a precondition that $P$ is contained in
   the boundaries of $k\ge 1$ pinched faces that are contained in $L$.
   During a recursive call, $P$ may have a charge $c\in\{0,1,2\}$. This
   charge should be at most 1 if $k\ge 1$, but can be 2 if $k=1$.

   To initialize the discharging procedure, we choose an arbitrary
   pinched face $f$ contained in $u$.  The face $f$ contains several
   chord paths $P_1,\ldots,P_r$, $r\ge 2$.  We discharge $f$ onto $P_1$
   and apply the recursive procedure to $P_1$, with a charge of 1 (with
   $f$ $L$ being the part of $u\setminus P_1$ that contains $f$).  We then
   recursively apply the discharging procedure on each of $P_2,\ldots,P_r$
   with a charge of 0.

   Next we describe each recursive step at which we are given $P$ with
	some charge $c\in\{0,1,2\}$.  There are several cases to consider (see \figref{discharging}):
	\begin{figure}
		\begin{center}
		\begin{tabular}{cc}
			\includegraphics{figs/discharge-A-2} &
			\includegraphics{figs/discharge-A-3} \\
			 2.a & 2.b \\[1.5em]
			\includegraphics{figs/discharge-A-4} &
			\includegraphics{figs/discharge-A-5} \\
			 2.c & 3 
		\end{tabular}
		\end{center}
		\caption{Discharging steps in the proof of \lemref{many-good}.}
		\figlabel{discharging}
	\end{figure}
  \begin{enumerate}
     \item $R$ contains no face of $\dual{T}$ that is pinched by $C$.
     This could occur if $R$ is empty, because $P$ is a keeper path,
     in which case we leave a charge of $c$ on it and we are done.
     Otherwise $R$ is non-empty and \lemref{one-caressed}  ensures that
     $R$ contains at least one caressed face $f$.  We move the charge
     from $P$ onto $f$ and we are done.

     \item $R$ contains a face $f$ that is pinched by $C$ and that shares
     at lest one edge with $P$.  We consider three subcases:
     \begin{enumerate}
        \item $f$ contains neither endpoint of $P$. In this case,
        $R\setminus f$ has at least three connected components, $A$,
        $B$, and $X_1,\ldots,X_k$, where $A$ and $B$ each contain an
        endpoint of $P$ and each $X_i$ has a chord path $P_i$ path in
        common with $f$.  We recurse on each of these components so
        that each of these components takes the place of $R$ in the
        recursion. When recursing on $A$ we take one unit of charge
        from $P$ (if needed).  When recursing on $B$ we take the second
        unit of charge from $P$ (if needed).  When recursing on $X_1$ we
        the unit of charge from $f$. When recursing on $X_2,\ldots,X_k$
        we use no additional charge.

	\item $f$ contains exactly one endpoint of $P$.  In this case,
	$u\setminus f$ has one connected component $A$ that contains an
	endpoint of $P$, one or more connected components $X_1,\ldots,X_k$
	where each $X_i$ has a chord path $P_i$ on the boundary of $f$.

	The path $P$ has a charge $c\le 2$.  When recursing on $X_1$ we
	assign all of $P$'s charge to the chord path $P_1$, which is contained
        in the single pinched face $f$.
		     
	When recursing on $A$ we assign move the single unit of charge
	from $f$ to the chord path of $A$.

        \item $f$ contains both endpoints of $P$.  In this case, $P$ must
        be on the boundary of several pinched faces in $L$. Therefore $P$
        has at most one unit of charge assigned to it.  Now, $R\setminus
        f$ has one or more connected components $X_1,\ldots,X_k$ sharing
        chord paths $P_1,\ldots,P_k$ with $f$ on which we recurse.
        When recursing on $X_1$ we move the charge from $P$ and the
        charge from $f$ to $P_1$.  When recursing on the remaining $X_i$,
        $i\in\{2,\ldots,k\}$ we assign no 	charge to $P_i$.
  \end{enumerate}
  \item $R$ contains at least one pinched face, but no pinched face in $R$
  shares an edge with $P$.  In this case, consider the face $g$ of $H$
  that is contained in $R$ and has $P$ on its boundary. By definition,
  $g$ contains no bad faces of $\dual{T}$, but $g$ is touched by $C$,
  so $g$ contains at least one good face\footnote{In face $g$ contains at
  least two good faces, one for each endpoint of $P$.} $f$ of $\dual{T}$.
  We move the $c$ units of charge from $P$ onto $f$.

  Now, $R$ still contains There are one or more pinched faces
  $f_1,\ldots,f_k$, where each $f_i$ shares a chord path $P_i$ with $g$.
  On each such face $f_i$, we run the initialization procedure described
  above except that we recurse only on the chord paths of $f_i$ that
  do not share edges with $g$. i.e., we do not recurse on the chord
  path $P_i$.
\end{enumerate}
   This completes the description of the discharging procedure, and the proof.
\end{proof}

\bibliographystyle{plain}
\bibliography{freecoll}

\end{document}









