\documentclass{patmorin}
%\usepackage[top=0.85in, bottom=0.85in, left=0.85in, right=0.85in]{geometry}

%%%%%%%%%%%%%%%%%%%%%%%%%%%%%%%%%%%%%%%%%%%%%%%%%%%%%%%%%%%%%%%%%%%%%%%%%%%%%%%%%%%%%%%%%%
%% Packages
%%%%%%%%%%%%%%%%%%%%%%%%%%%%%%%%%%%%%%%%%%%%%%%%%%%%%%%%%%%%%%%%%%%%%%%%%%%%%%%%%%%%%%%%%%
\usepackage{amsmath}
\usepackage{amsfonts}
\usepackage{amsthm}
\usepackage{graphicx}
%\usepackage{xspace}
%\usepackage{wrapfig}
\newenvironment{wrapfigure}[9]{\begin{figure}}{\end{figure}}
\newcommand{\Vspace}[1]{}
\usepackage{enumerate}
\usepackage{cite}
\usepackage{pat}
\usepackage{paralist}
\usepackage{hyperref}
\hypersetup{colorlinks=true, linkcolor=linkblue,  anchorcolor=linkblue,
citecolor=linkblue, filecolor=linkblue, menucolor=linkblue,
urlcolor=linkblue, pdfcreator=Me, pdfproducer=Me} 
\setlength{\parskip}{1ex}
\usepackage{algorithm}
\usepackage{subfig}
%\usepackage{subfigure}
\usepackage{array}
\usepackage[noend]{algpseudocode}
\usepackage[usenames]{xcolor}
\usepackage{stmaryrd} % for lightning-symbol
\usepackage{mathtools} % for mathclap in \conf-definition
\usepackage{todonotes}
%\usepackage{compress}
%\usepackage{times}
\usepackage{lineno}

\title{\MakeUppercase{Dual Cycles and Collinear Sets in Triangulations}%
    \thanks{This work was partly funded by NSERC and MRI.}}

\author{Vida Dujmovi\'c\thanks{Department of Computer Science and Electrical Engineering, University of Ottawa}\,\, and 
        Pat Morin\thanks{School of Computer Science, Carleton University}}

\newcommand{\dual}[1]{{#1}^\star}
\newcommand{\note}[2]{{\color{red}#1:~#2}}

\begin{document}
\maketitle


\begin{abstract}
   We show that, if a $n$-vertex triangulation $T$ of maximum degree
   $\Delta$ has a dual that contains a cycle of length $\ell$, then
   $T$ has a plane straight-line drawing in which some \emph{collinear
   set} of $\Omega(\ell/\Delta^2)$ vertices lie on a line.  Using the
   current lower bounds on the length of longest cycles in 3-regular
   3-connected graphs, this implies that $T$ has a collinear set of
   size $\Omega(n^{0.8}/\Delta^2)$.  Such collinear sets have numerous
   results in graph drawing and related areas.
\end{abstract}

\section{Introduction}

For a planar graph $G$, we say that a set $S\subseteq V(G)$ is
a \emph{collinear set} if $G$ has a non-crossing straight-line drawing in
which the vertices of $S$ are all collinear.  The \emph{dual} $\dual{G}$
of a plane graph $G$ is the graph whose vertex set $V(\dual{G})$ is
the set of faces in $G$ and in which $f,g\in E(\dual{G})$ if and only
if the face $f$ and the face $g$ have at least one edge in common.
The \emph{circumference}, $c(G)$ of a graph $G$ is the length of its
longest cycle. We prove the following theorem:

\begin{thm}\thmlabel{main}
  Let $T$ be a triangulation of maximum degree $\Delta$ whose dual
  $\dual{T}$ has circumference $\ell$. Then $T$ has a collinear set of
  size $\Omega(\ell/\Delta^2)$.
\end{thm}

The dual of a triangulation is a 3-connected cubic planar graph.
The study of the circumference of 3-connected cubic planar graphs
has a long and rich history going back at least 1884 when Tait
\cite{tait:listings},conjectured that every such graph is Hamiltonian.  In
1946, Tait's conjecture was disproved by Tutte who gave a non-Hamiltonian
46-vertex example \cite{tutte:on}.  Repeatedly replacing vertices of
this graph with copies of itself gives a family of graphs, $\langle G_i:i\in
\Z\rangle$ in which $G_i$ has $46\cdot 45^i$ vertices and circumference at
most $45\cdot44^i$.  Stated another way, $n$-vertex members of the
family have circumference $O(n^a)$, for $a=\log_{44}(45)) < n^{0.9941}$.
The current best upper bound of this type is due to Gr\"unbaum and
Walther \cite{grunbaum.walther:shortness} who construct a 24-vertex
non-Hamiltonian cubic 3-connected planar graph, resulting in a family
of graphs in which $n$-vertex members have circumference $O(n^{\alpha})$
for $\alpha=\log_{23}(22)< 0.9859$.

A series of results has steadily improved the lower bounds on the
circumference of $n$-vertex  (not necessarily planar) 3-connected cubic
graphs.  Barnette \cite{barnette:4} showed that, for every $n$-vertex
3-connected cubic graph $G$, $c(G)=\Omega(\log n)$.  Bondy and Simonovits
\cite{bondy.simonovits:7} improved this bound to $e^{\Omega(\sqrt{\log
n})}$ and conjectured that it can be improved to $\Omega(n^\alpha)$ for $\alpha>0$.
Jackson \cite{jackson:8} confirmed this bound with $\alpha=\log_2(1+\sqrt{5})-1
> 0.6942$.  Billinksi \etal\ \cite{billinksi.jacdson.ea:6} improved this
to the root of $4^{1/\alpha}-3^{1/\alpha}=2$, which implies $\alpha>0.7532$.  The current
record is held by Liu, Yu, and Zhang \cite{liu.yu.zhang:circumference}
who show that $\alpha>0.8$.  Together with \thmref{main}, this result implies
the following corollary:

\begin{cor}\corlabel{main}
  Every $n$-vertex triangulation of maximum degree $\Delta$ contains a
  collinear set of size $\Omega(n^{0.8}/\Delta^2)$.
\end{cor}

A longstanding open problem on circumference is \emph{Barnette's
Conjecture}, which asserts that every \emph{bipartite} 3-connected cubic
planar graph is Hamiltonian. Note that, if Barnette's conjecture is true,
this would imply that every triangulation in which every vertex has even
degree contains a collinear set of linear size.  \note{PM}{Think about
whether this even makes sense. For example, the assumption can only be
true when $n$ is odd.  Maybe there's a little trick we can do to make
it more generally applicable, like when only a small number, $k$, of
vertices have odd degree.}



\section{Proof of \thmref{main}}

Let $G$ be a plane graph.  We treat the vertices of $G$ as points,
the edges of $G$ as closed curves and the faces of $G$ as closed sets
(so that a face contains all the edges on its boundary and an edge
contains both its endpoints).  Whenever we consider subgraphs of $G$
we treat them as having the same embedding as $G$.  Similarly, if we
consider a graph $\bar{G}$ that is homeomorphic to $G$ then we assume
that the edges of $\bar{G}$---each of which repreesents a path in $G$
whose internal vertices all have degree 2---inherit their embedding from
the paths they represent in $G$.

Finally, if we consider the dual $G^*$ of $G$ then we treat it as a
plane graph in which each vertex $f$ is represented as a point in the
interior of the face $f$ of $G$ that it represents.  The edges of $G^*$
are embedded so that an edge $fg$ is contained in the union of the two
faces $f$ and $g$ of $G$, it intersects the interior of exactly one
edge of $G$ that is common to $f$ and $G$, and this intersection
consists of a single point.

A \emph{proper good curve} $C$ for a plane graph $G$ is a
Jordan curve with the following properties:
\begin{enumerate}
  \item proper: for any edge $xy$ of $G$, $C$ either contains $xy$, intersects
  $xy$ in a single point (possibly an endpoint) or $C$ is disjoint
  from $xy$; and
  \item good: $C$ contains at least one point in the interior of
  the outer face of $G$.
\end{enumerate}

Da Lozzo \etal\ show that proper good curves define collinear sets:

\begin{thm}
  In a plane graph $G$, a set $S\subseteq V(G)$ is a collinear set if
  and only if there is a proper good curve for $G$ that contains $S$.
\end{thm}

For a triangulation $T$, let $v(T)$ denote the size of the largest
collinear set in $T$.  We will show that, for any triangulation $T$
of maximum degree $\Delta$
whose dual is $T^*$, $v(T)=\Theta(c(\dual{T})/\Delta^2)$ by demonstrating a
relationship between proper good curves in $T$ and cycles in $\dual{T}$.

In one direction, this result is easy, as used by Ravsky and Verbitsky
\cite{ravsky.verbitsky:on,ravsky.verbitsky:on-arxiv}.  If $T$ is a
triangulation that has a proper good curve $C$ containing $k$ vertices,
then a slight deformation of $C$ produces a proper-good curve that
contains no vertices. This curve intersects a cyclic sequence of faces
$f_0,\ldots,f_{k'-1}$ of $T$ with $k'\ge k$.  In this sequence, $f_i$ and
$f_{(i+1)\bmod k'}$ share an edge, for every $i\in\{0,\ldots,k'-1\}$, so
this sequence is a closed walk in the dual $\dual{T}$ of $T$.  Property~1
of good curves and the fact that each face of $T$ is a triangle ensures
that $f_i\neq f_j$ for any $i\neq j$, so this sequence is a cycle in
the dual of length $k'\ge k$.  Therefore, $c(\dual{T})\ge v(T)$.

\subsection{Faces that are Touched, Pinched, and Good}

Let $C$ be a cycle in $\dual{T}$.  We say that $C$ is non-trivial if it
is not a single face of $\dual(T)$.   We say that a face $f$ of $\dual{T}$ 
\begin{enumerate}
  \item is \emph{touched} by $C$ if $f\cap C\neq \emptyset$;
  \item is \emph{pinched} by $C$ if $f\cap C$ has more than one connected component;
  \item is \emph{good} for $C$ if it is touched but not pinched by $C$.
\end{enumerate}

For a non-trivial cycle $C$ in $\dual{T}$, we use $\tau_{\dual{T}}(C)$,
$\rho_{\dual{T}}(C)$ and $\gamma_{\dual{T}}(C)$ to denote the set of
faces in $\dual{T}$ that are $\tau$ouched, $\rho$inched, and $\gamma$ood
for $C$, respectively.

\begin{lem}\lemlabel{cycle-to-curve}
   Let $T$ be a triangulation and $C$ be a non-trivial cycle in
   $\dual{T}$.  Then $T$ has a proper-good curve that contains at least
   $|\gamma(C)|/4$ vertices, i.e., $v(T)\ge |\gamma(C)|/4$.
\end{lem}

\begin{proof}
  Let $F$ be the set of faces in $\dual{T}$ that are good for $C$. Each
  element $u\in F$ corresponds to a vertex of $T$ so we will treat $F$
  as a set of vertices in $T$.  Consider the subgraph $T[F]$ of $T$
  induced by $F$.  This graph is planar and has $k$ vertices. Therefore,
  by the 4-Colour Theorem it contains an independent set $F'\subset F$
  of size at least $k/4$.

  We claim that $T$ has a proper-good curve that contains all the vertices
  in $F'$.  To see this, first observe that the cycle $C$ in $\dual{T}$
  already defines a proper-good curve (that does not contain any vertices
  of $T$) that we will also call $C$.  We will perform surgery on $C$
  so that it contains all the elements of $F'$.

  For each vertex $u\in F'$, let $w_0,\ldots,w_{d-1}$ denote the
  neighbours of $u$ in cyclic order.  The curve $C$ intersects some
  contiguous subsequence $uw_i,\ldots,uw_j$ of the edges adjacent
  to $u$.  Since $C$ is non-trivial, this sequence does not contain all
  edges incident to $u$. In particular, the curve $C$ crosses the edge
  $w_{i-1}w_i$, then crosses
  $uw_i,\ldots,uw_j$, and then crosses the edge $w_j w_{j+1}$.  We modify
  $C$ by removing the portion between the first and last of these crossings
  and replacing it with a curve that contains $u$ and is contained in the
  two triangles $w_{i-1}uw_i$ and $w_{j-1}uw_j$.

  After performing this surgery for each $u\in F'$ we have a curve $C'$
  that contains every vertex $u\in F'$.  All that remains is verify that
  $C'$ is good and proper for $T$. That $C'$ is good for $T$ is
  obvious.  That $C'$ is proper for $T$ follows the following two observations:
  (i)~$C'$ does not contain any two adjacent vertices (since $F'$ is an
  independent set); and (ii)~if $C'$ contains a vertex $u$, then it does
  not intersect the interior of any edge incident to $u$.
\end{proof}

\lemref{cycle-to-curve} reduces the problem to finding a cycle in
$\dual{T}$ that is good for many faces.  We do this by showing that any
long cycle in $\dual{T}$ can be transformed into a cycle that is good
for many good faces of $\dual{T^*}$.  To do this, we will introduce an
auxilliary graph.

\begin{lem}\lemlabel{one-good}
   Let $T$ be a triangulation, $C$ be a non-trivial cycle in $\dual{T}$,
   and let $P$ be a path in $\dual{T}$ whose endpoints in $V(C)$ and whose
   internal vertices are in the interior of $C$.  Then each of the two
   internal faces of $P\cup C$ contains at least one face of $\dual{T}$
   that is good for $C$.
\end{lem}

\begin{proof}
   It suffices to consider one of the two faces in question. Let $R$
   be the face of $P\cup C$ to the right of $P$.  The proof
   is by induction on the number, $t$, of faces of $\dual{T}$ contained
   in $R$.  If $t=1$, then $R$ is a face of $\dual{T}$ and it is good
   for $C$.

   If $t>1$, then consider the face $f$ of $\dual{T}$ that is contained
   in $R$ and has the first edge of $P$ on its boundary.  Since $t>1$,
   $X=R\setminus f$ is non-empty. The set $X$ may have several connected
   components, but each of them has a boundary that contains a path $P'$
   whose endpoints are vertices of $C$ and whose interior vertices are
   in the interior of $C$.  We can therefore apply induction on the
   path $P'$.
\end{proof}

Let $C$ be a non-trivial cycle in $\dual{T}$ and consider the auxilliary
graph $H$ with vertex set $V(H)\subseteq V(\dual{T})$ and whose edge set
consist of the edges of $C$ plus those edges of $\dual{T}$ that belong
to any face pinched by $C$.  Let $v_0,\ldots,v_{r-1}$ be the cyclic sequence of vertices on some face $f$ of $\dual{T}$ that is pinched by $C$.  
We identify two kinds of vertices that are \emph{special} with respect to $f$:
\begin{enumerate}
  \item A vertex $v_i$ is special of \emph{Type~A} if $v_{i-1}v_i$ is an edge of $C$ and $v_iv_{i+1}$ is not an edge of $C$.
  \item A vertex $v_i$ is special of \emph{Type~B} if $v_{i-1}v_i$ is not an edge of $C$ and $v_iv_{i+1}$ is an edge of $C$.
  \item A vertex $v_i$ is special of \emph{Type~Y} if $v_i$ not incident to any edge of $C$ and $v_i$ has degree 3 in $H$.
\end{enumerate}

We say that a path $v_i,\ldots,v_j$ is a \emph{keeper} with respect to
$f$ if $v_i$ is special of Type~A, $v_j$ is special of Type~B, and none
of $v_{i+1},\ldots,v_{j-1}$ are special.  We let $\tilde{H}$ denote the
subgraph of $H$ containing all the edges of $C$ and all the edges of
all paths that are special with respect to some face $f$ of $\dual{T}$.

It is worth remarking at this point that, by definition, every keeper
is contained in the boundary of at least one face $f$ of $\dual{T}$.
This property will be useful shortly.

Let $\bar{H}$ denote the graph that is homeormophic to $\tilde{H}$ but does not
contain any degree 2 vertices.  That is, $\bar{H}$ is the minor of $\tilde{H}$
obtained by repeatedly contracting an edge incident a degree-2 vertex.
The graph $\bar{H}$ naturally inherits an embedding from the embedding of $\tilde{H}$ (which inherits its embedding from an embedding $H$, which inherits an embedding from $\dual{T})$.  This embedding partitions the edges of $\bar{H}$ into three sets:
\begin{enumerate}
  \item The set $B$ of edges that are contained in (the embedding of) $C$;
  \item The set $E_0$ of edges whose interiors are contained in the interior of (the embedding of) $C$; and
  \item The set $E_1$ of edges whose interiors are contained in the exterior of (the embedding of) $C$.
\end{enumerate}

Observe that, for each $i\in\{0,1\}$, the graph $H_i$ whose edges are
exactly those in $B\cup E_i$ is outerplanar.  Let $T_i$ be the subgraph of
$\dual{H_i}$ whose edges are all those dual to the edges of $E_i$. From
the outerplanarity of $H_i$, it follows that $T_i$ is a tree. Note that
the nodes of $T_i$, which are faces of $\bar{H}$, are subsets of the
plane obtained by taking the closure of the union of faces in $\dual{T}$.
In the following, when we say that a node $u$ of $T_i$ contains a face
$f$ of $\dual{T}$ we mean that $f$ is one of the faces of $\dual{T}$
whose union makes up $u$.

\note{PM: The following is bullshit! A node u doesn't have Property~Y.}
For a node $u$ in $T_i$, let $C_u$ denote the cycle in $\dual{T}$ that
traverses the boundary of $u$.  Such a cycle has a special property that
we call \emph{Property~Y:} Let $f$ be any face of $\dual{T}$ that is
pinched by $C_u$ and contained in the interior of $C_u$ and let $P$ be any
path on the boundary of $f$ whose two endpoints are in $V(C)$ and whose
interior vertices are disjoint from $V(C)$.  Then $P$ has at least one
interior vertex $v$ that is on the boundary of two or more bad faces. In
particular, one of these faces is a bad face that is distinct from $f$.

\begin{lem}\lemlabel{all-good}
   Let $C_u$ be a cycle in $\dual{T}$ with Property~Y and let $u$ be
   region bounded by $C_u$.  Then $\gamma(C_u)\cap u= \tau(C_u)\cap u$.
\end{lem}

\begin{proof}
   Suppose, for the sake of contradiction, that $u$ contains some face
   $f$ that is pinched by $C_u$.  The face $f$ contains a chord-path
   $P$ for $C_u$. The path $P$ partitions $u$ into two sets $u'$ and
   $u''$ and assume, without loss of generality that $u''$ contains $f$.
   By Property~Y, $P$ contains an interior vertex that is incident to two
   pinched faces, one of which, $f'$ is not $f$ and is therefore contained
   in $u'$.  Now, since $f'$ is pinched, it contains a chord path $P'$
   for $C_u$ that is disjoint from $f$.  Exactly the same reasoning
   applied to $P'$ and $f'$ proves the existence of a third bad face $f''$
   and a third chord path $P''$ that is disjoint from $f$ and $f'$.
   Repeating this argument \emph{ad infinitum} contradicts the assumption
   that $\dual{T}$ is finite.
\end{proof}

To avoid repeated awkward phraseology, the next few lemmas prove
statements about $T_0$.  However, exactly the same statements apply
to $T_1$ since, after an appropriate inversion of the plane, the same
proofs apply.

The following lemma allows us to direct our effort towards proving that
$T_0$ has many leaves.

\begin{lem}
   Each leaf $u$ of $T_0$ contains at least one face of $\dual{T}$
   that is good for $C$.
\end{lem}

\begin{proof}
   Without loss of generality, assume $i=0$.  The edge of $T_i$ incident
   to $u$ corresponds to a chord path $P$. The graph $P\cup C$ has two
   internal faces, one of which is $u$.  The lemma now follows immediately
   from \lemref{one-good}.
\end{proof}

The next lemma shows that a node of $T_0$ that contains many touched
faces must have high degree or contain many good faces.
\begin{lem}
   For each node $u$ of $T_0$,  $\deg_{T_i}(u)+|\gamma(C)\cap u|\ge
   \beta|\tau(C)\cap u|$.
\end{lem}

\begin{proof}
   The boundary of $u$ consists of paths that alternate between
   \emph{black} paths whose edges are all in $E(C)$ and \emph{red}
   paths whose endpoints are in $V(C)$ but whose edges are not in
   $E(C)$.  The value of $\deg_{T_i}(u)$ is the number of red paths.
   Let $v_0,\ldots,v_{2d-1}$ denote the sequence of vertices on the
   boundary of $u$ that are are the common endpoint of a red and a black
   path, labelled so that the path from $v_0$ to $v_1$ is a red path.

   We will consider a sequence of operations that transforms $C$.  Each
   such operation removes the path in $C$ that with endpoints $v_{2i}$
   and $v_{2i+1}$ that has no edges in the boundary of $u$.  Let $Q_i$
   be the red path with endpoints $v_{2i}$ and $v_{2i+1}$ and note that,
   by construction, there is a single face $f$ of $\dual{T}$ that contains
   $Q$ on its boundary.   There are then two cases to consider:
   \begin{enumerate}
       \item The face $f$ is contained in $u$.  In this case, we complete
       $C$ by adding the edges of $Q_i$ to $C$.
       \item The face $f$ is not contained in $u$. In this case,
       we complete $C$ by adding the path on $f$'s boundary that has
       $v_{2i}$ and $v_{2i+1}$ as endpoints but is not $Q_i$.
   \end{enumerate}

   Note that, in either case, this operation increases the number of faces
   of $u$ that are good for $C$ by at most 1.  This occurs precisely when,
   in the first case, $f\cap C$ has exactly two components. 
   
   By \lemref{all-good} we know that $u$ contains faces that are pinched
   by $C_u$.  What about the faces that are pinched by $C'$. 

 e
   We know that $C_u$ contains no faces pinched by $C_u$.  What about faces pinched by $C'$?  
   We know that $u$ contains no faces that are bad for $C_u$

   Therefore, after performing this operation on each red path $Q_i$,
   we are left with a cycle $C'$ that contains $u$ and.
   \[
       |\gamma(C')\cap u| \le |\gamma(C)\cap u| + \deg_{T_i}(u) \enspace .
   \]
   Now, we can think of $C'$ as being created from the boundary of $u$ by gluing at most $\deg_{T_i}(u)$ faces to $u$.  Each such gluing operation introduces a new face and this face is good for $C'$.  At the same time 

 plus at most $\deg_{T_i}(u)$ additional faces.  This implies that the number of faces in the interior of $C'$ is not more than the number of faces in $u$ that are good for $C$ minus the $2\deg_{T_i}(u)$.

\end{proof}




\begin{thm}
  Let $T$ be a triangulation of maximum degree $\Delta$ whose dual
  $\dual{T}$ has a non-trivial cycle $C$ of length $\ell$.  Then
  $\dual{T}$ has a cycle $C'$ that is good for $\Omega(\ell/\Delta^2)$
  faces of $\dual{T}$.
\end{thm}


\begin{lem}

\end{lem}

\begin{proof}

\end{proof}



\section{}

\newpage
\bibliographystyle{plain}
\bibliography{freecoll}

\end{document}









