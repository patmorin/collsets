\documentclass[a4paper,UKenglish]{socg-lipics-v2019}
%\usepackage[top=0.85in, bottom=0.85in, left=0.85in, right=0.85in]{geometry}
\listfiles
%%%%%%%%%%%%%%%%%%%%%%%%%%%%%%%%%%%%%%%%%%%%%%%%%%%%%%%%%%%%%%%%%%%%%%%%%%%%%%%%%%%%%%%%%%
%% Packages
%%%%%%%%%%%%%%%%%%%%%%%%%%%%%%%%%%%%%%%%%%%%%%%%%%%%%%%%%%%%%%%%%%%%%%%%%%%%%%%%%%%%%%%%%%
\usepackage[T1]{fontenc}
\usepackage[utf8]{inputenc}
\usepackage{amsmath}
\usepackage{amsfonts}
\usepackage{amsthm}
\usepackage{graphicx}
%\usepackage{xspace}
%\usepackage{wrapfig}
\newcommand{\Vspace}[1]{}
\usepackage{enumerate}
\usepackage{cite}
\usepackage{pat}
\usepackage{paralist}
\usepackage{hyperref}
%\hypersetup{colorlinks=true, linkcolor=linkblue,  anchorcolor=linkblue,
%citecolor=linkblue, filecolor=linkblue, menucolor=linkblue,
%urlcolor=linkblue, pdfcreator=Me, pdfproducer=Me} 
\setlength{\parskip}{1ex}
\usepackage{algorithm}
%\usepackage{subfig}
\usepackage{array}
%\usepackage[noend]{algpseudocode}
\usepackage[usenames]{xcolor}
%\usepackage{compress}
%\usepackage{times}
%\usepackage{lineno}

%\bibliographystyle{plainurl}

\title{Dual Circumference and Collinear Sets}

\author{Vida Dujmovi\'c}
       {School of Computer Science and Electrical Engineering, 
        University of Ottawa, Canada \and \url{https://cglab.ca/~vida/}}
       {vida.dujmovic@uottawa.ca}
       {} % orcid url
       {} % author specific funding

\author{Pat Morin}
       {School of Computer Science,
        Carleton University, Canada \and \url{https://cglab.ca/~morin/}}
       {morin@scs.carleton.ca}
       {} % orcid url
       {} % author specific funding

\authorrunning{V. Dujmovi\'c and P. Morin}

\keywords{Planar graphs, collinear sets, untangling, column planarity,
          universal point subsets, partial simultaneous geometric drawings}
\Copyright{Vida Dujmovi\'c and Pat Morin}
\ccsdesc{Mathematics of computing~Graph theory}
\ccsdesc{Mathematics of computing~Extremal graph theory}
\ccsdesc{Human-centered computing~Graph drawings}
\relatedversion{A full version of this paper is available at 
                \url{https://arxiv.org/abs/1811.03427}}
\funding{This work was partly funded by NSERC and MRI.}

%% Event-specific LiPiCS stuff
\EventEditors{Gill Barequet and Yusu Wang}
\EventNoEds{2}
\EventLongTitle{35th International Symposium on Computational Geometry (SoCG 2019)}
\EventShortTitle{SoCG 2019}
\EventAcronym{SoCG}
\EventYear{2019}
\EventDate{June 18--21, 2019}
\EventLocation{Portland, Oregon, USA}
\EventLogo{socg-logo}
\SeriesVolume{129}
\ArticleNo{29}  

\acknowledgements{Much of this research took place during the Sixth
  Workshop on Order and Geometry held in Ciążeń, Poland, September
  19--22, 2018.  The authors are grateful to the organizers,
  Stefan~Felsner and Piotr~Micek, and to the other participants for
  providing a stimulating research environment.}

\nolinenumbers

\newcommand{\dual}[1]{{#1}^\star}
%\newcommand{\note}[2]{{\color{red}[#1:~#2]}}

\begin{document}
\maketitle


\begin{abstract}
   We show that, if a $n$-vertex triangulation $T$ of maximum degree
   $\Delta$ has a dual that contains a cycle of length $\ell$, then $T$
   has a non-crossing straight-line drawing in which some \emph{collinear
   set} of $\Omega(\ell/\Delta^4)$ vertices lie on a line.  Using the
   current lower bounds on the length of longest cycles in 3-regular
   3-connected graphs, this implies that every $n$-vertex planar
   graph of maximum degree $\Delta$ has a collinear set of size
   $\Omega(n^{0.8}/\Delta^4)$.  Very recently, Dujmovi\'c \etal\ (SODA
   2019) showed that, if $S$ is a collinear set in a triangulation $T$
   then, for any point set $X\subset\R^2$ with $|X|=|S|$, $T$ has a
   non-crossing straight-line drawing in which the vertices of $S$ are
   drawn on the points in $X$.  Because of this, collinear sets have
   numerous applications in graph drawing and related areas.
\end{abstract}

\section{Introduction}

Throughout this paper, all graphs are simple and finite and have at least
4 vertices.  For a planar graph $G$, we say that a set $S\subseteq V(G)$
is a \emph{collinear set} if $G$ has a non-crossing straight-line drawing
in which the vertices of $S$ are all collinear.  A \emph{plane graph}
is a planar graph $G$ along
with a particular non-crossing drawing of $G$.  The \emph{dual} $\dual{G}$
of a plane graph $G$ is the graph whose vertex set $V(\dual{G})$
is the set of faces in $G$ and in which $fg\in E(\dual{G})$ if and
only if the faces $f$ and $g$ of $G$ have at least one edge in common.
The \emph{circumference}, $c(G)$, of a graph $G$ is the length of its
longest cycle. In \secref{proof}, we prove the following theorem:

\begin{thm}\thmlabel{main}
  Let $T$ be a triangulation of maximum degree $\Delta$ whose dual
  $\dual{T}$ has circumference $\ell$. Then $T$ has a collinear set of
  size $\Omega(\ell/\Delta^4)$.
\end{thm}

The dual of a triangulation is a 3-connected cubic planar graph.
The study of the circumference of 3-connected cubic planar graphs
has a long and rich history going back to at least 1884 when Tait
\cite{tait:remarks} conjectured that every such graph is Hamiltonian.  In
1946, Tait's conjecture was disproved by Tutte who gave a non-Hamiltonian
46-vertex example \cite{tutte:on}.  Repeatedly replacing vertices of
Tutte's graph with copies of itself gives a family of graphs, $\langle G_i:i\in
\Z\rangle$ in which $G_i$ has $46\cdot 45^i$ vertices and circumference at
most $45\cdot44^i$.  Stated another way, $n$-vertex members of the
family have circumference $O(n^a)$, for $a=\log_{44}(45) < 0.9941$.
The current best upper bound of this type is due to Gr\"unbaum and
Walther \cite{grunbaum.walther:shortness} who construct a 24-vertex
non-Hamiltonian cubic 3-connected planar graph, resulting in a family
of graphs in which $n$-vertex members have circumference $O(n^{\alpha})$
for $\alpha=\log_{23}(22)< 0.9859$.

A series of results has steadily improved the lower bounds on the
circumference of $n$-vertex  (not necessarily planar) 3-connected
cubic graphs.  Barnette \cite{barnette:trees} showed that, for
every $n$-vertex 3-connected cubic graph $G$, $c(G)=\Omega(\log n)$.
Bondy and Simonovits \cite{bondy.simonovits:longest} improved this bound
to $e^{\Omega(\sqrt{\log n})}$ and conjectured that it can be improved
to $\Omega(n^\alpha)$ for some $\alpha>0$.  Jackson \cite{jackson:longest}
confirmed this conjecture with $\alpha=\log_2(1+\sqrt{5})-1 > 0.6942$.
Billinksi \etal\ \cite{bilinksi.jackson.ea:circumference} improved
this to the solution of $4^{1/\alpha}-3^{1/\alpha}=2$, which implies
$\alpha>0.7532$.  The current record is held by Liu, Yu, and Zhang
\cite{liu.yu.zhang:circumference} who show that $\alpha>0.8$.  

It is known that any planar graph of maximum degree $\Delta$ can be
triangulated so that the resulting triangulation has maximum degree
$\lceil 3\Delta/2\rceil+11$ \cite{kant.bodlaender:triangulating}. This
fact, together with \thmref{main} and the result of Liu, Yu, and Zhang
\cite{liu.yu.zhang:circumference}, implies the following corollary:

\begin{cor}\corlabel{main}
  Every $n$-vertex triangulation of maximum degree $\Delta$ contains a
  collinear set of size $\Omega(n^{0.8}/\Delta^4)$.
\end{cor}

%A longstanding open problem on circumference is \emph{Barnette's
%Conjecture}, which asserts that every \emph{bipartite} 3-connected cubic
%planar graph is Hamiltonian. Note that, if Barnette's conjecture is true,
%this would imply that every triangulation in which every vertex has even
%degree contains a collinear set of linear size.  \note{PM}{Think about
%whether this even makes sense. For example, the assumption can only be
%true when $n$ is odd.  Maybe there's a little trick we can do to make
%it more generally applicable, like when only a small number, $k$, of
%vertices have odd degree.}

It is known that every planar graph $G$ has a collinear set of size
$\Omega(\sqrt{n})$ \cite{bose.dujmovic.ea:polynomial,dujmovic:utility}
. \corref{main} therefore improves on this bound for bounded-degree planar
graphs and, indeed for the family of $n$-vertex planar graphs of maximum
degree $\Delta\in O(n^{\delta})$, with $\delta < 0.075$.  For example,
the graphs generated by Gr\"unbaum and Walther's construction have maximum
degree $\Delta \in O(\log n)$.  Therefore, there exist $n$-vertex planar
graphs of maximum degree $O(\log n)$ whose largest collinear set has size
$O(n^{0.9859})$.  \corref{main} implies that every
$n$-vertex planar graph of maximum degree $O(\log n)$ has a collinear
set of size $\Omega(n^{0.8})$.

Recently, Dujmovi\'c \etal\ \cite{dujmovic.frati.ea:every} have shown
that every collinear set is \emph{free}. That is, for any planar graph
$G$, any collinear set $S\subseteq V(G)$, and any set $X\subset\R^2$
with $|X|=|S|$, there exists a non-crossing straight-line drawing of $G$
in which the vertices of $S$ are drawn on the points of $X$.  Because of
this, collinear sets have immediate applications in graph drawing and
related areas.  For applications of \corref{main}, including untangling
\cite{cibulka:untangling,pach.tardos:untangling,watanabe:open,goaoc.kratochvil.ea:untangling,kang.pikhurko.ea:untangling,bose.dujmovic.ea:polynomial,dalozzo.dujmovic.ea:drawing,dujmovic:utility,ravsky.verbitsky:on},
column
planarity~\cite{barba.evans.ea:column,evans.kusters.ea:column,dalozzo.dujmovic.ea:drawing,dujmovic:utility},
universal point
subsets~\cite{digiacomo.liotta.ea:how,angelini.binucci.ea:universal,dalozzo.dujmovic.ea:drawing,dujmovic:utility},
and partial simultaneous geometric
drawings~\cite{evans.kusters.ea:column,barba.hoffmann.ea:column,angelini.evans.ea:sefe,blasius.kobourov.ea:simultaneous,dujmovic:utility}
the reader is referred to Dujmovi\'c \cite{dujmovic:utility}
and Dujmovi\'c \etal\ \cite[Section~1.1]{dujmovic.frati.ea:every}.
\corref{main} gives improved bounds for all of these problems for planar
graphs of maximum $\Delta\in o(n^{0.075})$.

For example, it is known that every $n$-vertex planar geometric
graph can be untangled while keeping $\Omega(n^{0.25})$ vertices fixed
\cite{bose.dujmovic.ea:polynomial} and that there are $n$-vertex planar
geometric graphs that cannot be untangled while keeping $O(n^{0.4948})$
vertices fixed \cite{cano.toth.ea:upper}. Although asymptotically
tight bounds are known for paths \cite{cibulka:untangling},
trees \cite{goaoc.kratochvil.ea:untangling}, outerplanar graphs
\cite{goaoc.kratochvil.ea:untangling}, planar graphs of treewidth
two \cite{ravsky.verbitsky:on}, and planar graphs of treewidth three
\cite{dalozzo.dujmovic.ea:drawing}, progress on the general case has
been stuck for 10 years due to the fact that the exponent $0.25$ comes
from two applications of Dilwerth's Theorem.  Thus, some substantially
new idea appears to be needed. By relating collinear/free sets to
dual circumference, the current paper presents an effective new idea.
Indeed, \corref{main} implies that every bounded-degree $n$-vertex
planar geometric graph can be untangled while keeping $\Omega(n^{0.4})$
vertices fixed.  Note that, even for bounded-degree planar graphs,
$\Omega(n^{0.25})$ was the best previously-known lower bound.

Our work opens two avenues for further progress:

\begin{enumerate}
   \item Lower bounds on the circumference of 3-regular
      3-connected graphs is an active area of research. Indeed,
      the $\Omega(n^{0.8})$ lower bound of Liu, Yu, and Zhang
      \cite{liu.yu.zhang:circumference} is less than a year old.  Any
      further progress on these lower bounds will translate immediately
      to an improved bound in \corref{main} and all its applications.

   \item It is possible that the dependence on $\Delta$ can be removed
      from \thmref{main} and \corref{main}, thus making these results
      applicable to all planar graphs, regardless of maximum degree.
\end{enumerate}

\section{Proof of \thmref{main}}
\seclabel{proof}

Let $G$ be a plane graph.  We treat the vertices of $G$ as points,
the edges of $G$ as closed curves and the faces of $G$ as closed sets
(so that a face contains all the edges on its boundary and an edge
contains both its endpoints).  Whenever we consider subgraphs of $G$
we treat them as having the same embedding as $G$.  Similarly, if we
consider a graph $G'$ that is homeomorphic to $G$ then we assume
that the edges of $G'$---each of which represents a path in $G$
whose internal vertices all have degree 2---inherit their embedding from
the paths they represent in $G$.

Finally, if we consider the dual $\dual{G}$ of $G$ then we treat it
as a plane graph in which each vertex $f$ is represented as a point in
the interior of the face $f$ of $G$ that it represents.  The edges of
$\dual{G}$ are embedded so that an edge $fg$ is contained in the union of
the two faces $f$ and $g$ of $G$, it intersects the interior of exactly
one edge of $G$ that is common to $f$ and $g$, and this intersection
consists of a single point.

A \emph{proper good curve} $C$ for a plane graph $G$ is a
Jordan curve with the following properties:
\begin{enumerate}
	\item[\emph{proper}:] for any edge $xy$ of $G$, $C$ either contains $xy$, intersects
  $xy$ in a single point (possibly an endpoint), or is disjoint
  from $xy$; and
  \item[\emph{good}:] $C$ contains at least one point in the interior of
  some face of $G$.
\end{enumerate}

Da Lozzo \etal\ \cite{dalozzo.dujmovic.ea:drawing} show that proper good
curves define collinear sets:

\begin{thm}\thmlabel{dalozzo}
  In a plane graph $G$, a set $S\subseteq V(G)$ is a collinear set if
  and only if there is a proper good curve for $G$ that contains $S$.
\end{thm}

For a triangulation $T$, let $v(T)$ denote the size of the largest
collinear set in $T$.  We will show that, for any triangulation $T$
of maximum degree $\Delta$
whose dual is $\dual{T}$, $v(T)=\Theta(c(\dual{T})/\Delta^4)$ by relating proper good curves in $T$ to cycles in $\dual{T}$.

As shown by Ravsky and Verbitsky
\cite{ravsky.verbitsky:on,ravsky.verbitsky:on-arxiv}, the inequality $v(T)
\le c(\dual{T})$ is easy: If $T$ is a triangulation that has a proper
good curve $C$ containing $k$ vertices, then a slight deformation of
$C$ produces a proper good curve that contains no vertices. This curve
intersects a cyclic sequence of faces $f_0,\ldots,f_{k'-1}$ of $T$
with $k'\ge k$.  In this sequence, $f_i$ and $f_{(i+1)\bmod k'}$ share
an edge, for every $i\in\{0,\ldots,k'-1\}$, so this sequence is a closed
walk in the dual $\dual{T}$ of $T$.  The properness of the original curve
and the fact that each face of $T$ is a triangle ensures that $f_i\neq
f_j$ for any $i\neq j$, so this sequence is a cycle in $\dual{T}$ of
length $k'\ge k$.  Therefore, $c(\dual{T})\ge v(T)$. From the result
of Gr\"unbaum and Walther described above, this implies that there are
$n$-vertex triangulations $T$ such that $v(T) = O(n^{0.9859})$.

The other direction, lower-bounding $v(T)$ in terms $c(\dual{T})$
is more difficult. Not every cycle $C$ of length $\ell$ in $\dual{T}$
can be easily transformed into a proper good curve containing a similar
number of vertices in $C$.  In the next section, we describe three
parameters $\tau$, $\rho$, and $\kappa$ of a cycle $C$ in $\dual{T}$
and show that $C$ can always be transformed into a proper good curve
containing $\Omega(\kappa)$ vertices of $T$.

\subsection{Faces that are Touched, Pinched, and Caressed}

Throughout the remainder of this section, $T$ is a triangulation
whose dual is $\dual{T}$ and $C$ is a cycle in $\dual{T}$.
Refer to \figref{touched-pinched-caressed} for the following definitions.
We say that a face $f$ of $\dual{T}$
\begin{enumerate}
  \item is \emph{touched} by $C$ if $f\cap C\neq \emptyset$;
  \item is \emph{pinched} by $C$ if $f\cap C$ is a cycle or has more than 
    one connected component; and
  \item is \emph{caressed} by $C$ if it is touched but not pinched by $C$.
\end{enumerate}

\begin{figure}
  \begin{center}
    \includegraphics{figs/t0t1-2}
  \end{center}
  \caption{Faces of $\dual{T}$ that are pinched and caressed by $C$. $C$
  is bold, caressed faces are teal, pinched faces are pink, and untouched
  faces are unshaded.} \figlabel{touched-pinched-caressed}
\end{figure}


Since $C$ is almost always the cycle of interest, we will usually say
that a face $f$ of $\dual{T}$ is touched, pinched, or caressed, without
specifically mentioning $C$.  We will frequently use the values $\tau$,
$\rho$, and $\kappa$ to denote the number of faces of $\dual{T}$ in
some region that are $\tau$ouched, $\rho$inched or $\kappa$aressed.
Observe that, since every face that is touched is either pinched or
caressed, we have the identity $\tau = \rho + \kappa$.

\begin{lem}\lemlabel{cycle-to-curve}
   If $C$ caresses $\kappa$ faces of $\dual{T}$ then $T$ has a proper
   good curve that contains at least $\kappa/4$ vertices so, by
   \thmref{dalozzo}, $v(T)\ge \kappa/4$.
\end{lem}

\begin{proof}
  Let $F$ be the set of faces in $\dual{T}$ that are caressed by $C$. Each
  element $u\in F$ corresponds to a vertex of $T$ so we will treat $F$ as
  a set of vertices in $T$.  Consider the subgraph $T[F]$ of $T$ induced
  by $F$.  The graph $T[F]$ is planar and has $\kappa$ vertices. Therefore,
  by the 4-Colour Theorem \cite{robertson.seymour.ea:four-colour}, $T[F]$
  contains an independent set $F'\subseteq F$ of size at least $\kappa/4$.

  We claim that there is a proper good curve for $T$ that contains all
  the vertices in $F'$.  To see this, first observe that the cycle $C$ in
  $\dual{T}$ already defines a proper good curve (that does not contain
  any vertices of $T$) that we also call $C$.  We perform
  local modifications on $C$ so that it contains all the vertices in $F'$.

  For any vertex $u\in F'$, let $w_0,\ldots,w_{d-1}$ denote the neighbours
  of $u$ in cyclic order.  The curve $C$ intersects some contiguous
  subsequence $uw_i,\ldots,uw_j$ of the edges adjacent to $u$.  Since $u$
  is caressed, this sequence does not contain all edges incident
  to $u$. Therefore, the curve $C$ crosses the edge $w_{i-1}w_i$, then
  crosses $uw_i,\ldots,uw_j$, and then crosses the edge $w_j w_{j+1}$.
  We modify $C$ by removing the portion between the first and last of
  these crossings and replacing it with a curve that contains $u$ and is
  contained in the two triangles $w_{i-1}uw_i$ and $w_juw_{j+1}$. (See
  \figref{cycle-to-curve}.)

  \begin{figure}
     \begin{center}
	\begin{tabular}{cc}
		\includegraphics{figs/cycle-to-curve-1} &
		\includegraphics{figs/cycle-to-curve-2}
	\end{tabular}
     \end{center}
     \caption{Transforming the dual cycle $C$ into a proper good curve $C'$ containing $u$.}
	  \figlabel{cycle-to-curve}
  \end{figure}

  After performing this local modification for each $u\in F'$ we have
  a curve $C'$ that contains every vertex $u\in F'$.  All that remains
  is verify that $C'$ is good and proper for $T$. That $C'$ is good for
  $T$ is obvious.  That $C'$ is proper for $T$ follows the following two
  observations: (i)~$C'$ does not contain any two adjacent vertices (since
  $F'$ is an independent set); and (ii)~if $C'$ contains a vertex $u$,
  then it does not intersect the interior of any edge incident to $u$.
\end{proof}

\lemref{cycle-to-curve} reduces our problem to finding a cycle in
$\dual{T}$ that caresses many faces.  It is tempting to hope that
any sufficiently long cycle in $\dual{T}$ caresses many faces, but
this is not true; \figref{few-caressed} shows that even a Hamiltonian
cycle $C$ in $\dual{T}$ may caress only four faces, two inside $C$ and
two outside of $C$.  In this example, there is an obvious sequence of
faces $f_0,\ldots,f_k$, all contained in the interior of $C$ where $f_i$
shares an edge with $f_{i+1}$ for each $i\in\{1,\ldots,k\}$.  The only
faces caressed by $C$ are the endpoints $f_0$ and $f_k$ of this sequence.

\begin{figure}
   \begin{center}
       \includegraphics{figs/two-caressed}
   \end{center}
   \caption{A Hamiltonian cycle $C$ in $\dual{T}$ that caresses only four faces.}
   \figlabel{few-caressed}
\end{figure}

Our strategy is to define a tree structure, $T_0$ on groups of faces
contained in the interior of $C$ and a similar structure, $T_1$ on groups
of faces in the exterior of $C$.  We will then show that every leaf of
$T_0$ or $T_1$ contains a face caressed by $C$. In \figref{few-caressed},
the tree $T_0$ is the path $f_0,\ldots,f_k$ and, indeed, the leaves $f_0$
and $f_k$ of this tree are caressed by $C$.  After a non-trivial amount
of analysis of the trees $T_0$ and $T_1$, we will eventually show that,
if $C$ does not caress many faces, then $T_0$ and $T_1$ have many nodes,
but few leaves.  Therefore $T_0$ and $T_1$ have many degree-2 nodes.
This abundance of degree-2 nodes makes it possible to perform a surgery
on $C$ that increases the number of caressed faces.  Performing this
surgery repeatedly will then produce a curve $C$ that caresses many faces.

The following useful lemma shows that the example in \figref{few-caressed}
is tight.  A path $P=v_1,\ldots,v_r$ in $\dual{T}$ is a \emph{chord
path} (for $C$) if $v_1,v_r\in V(C)$ and $v_2,\ldots,v_{r-1}\not\in
V(C)$.  Note that this definition implies that the interior vertices
$v_2,\ldots,v_{r-1}$ of $P$ are either all contained in the interior of
$C$ or all contained in the exterior of $C$.

\begin{lem}\lemlabel{one-caressed}
   Let $P$ be a chord path for $C$ and let $L$ and $R$ be the two faces
   of $P\cup C$ that each contain $P$ in their boundary. Then $R$
   contains at least one face of $\dual{T}$ that is caressed by $C$.
\end{lem}

\begin{proof}
   The proof is by
   induction on the number, $t$, of faces of $\dual{T}$ contained in $R$.
   If $t=1$, then $R$ is a face of $\dual{T}$ and it is caressed by $C$.

   If $t>1$, then consider the face $f$ of $\dual{T}$ that is contained
   in $R$ and has the first edge of $P$ on its boundary.  Refer to
   \figref{one-caressed}. Since $t>1$, $X=R\setminus f$ is non-empty. The
   set $X$ may have several connected components $X_1,\ldots,X_k$, but
   each $X_i$ has a boundary that contains a chord path $P_i$ for $C$.
   We can therefore apply induction on $P_1$ (or any $P_i$) using $R=X_1$
   in the inductive hypothesis.
  \begin{figure}
     \begin{center}
	\begin{tabular}{cc}
		\includegraphics{figs/one-caressed-1} &
		\includegraphics{figs/one-caressed-2}
	\end{tabular}
     \end{center}
	  \caption{The proof of \lemref{one-caressed}.}
	  \figlabel{one-caressed}
  \end{figure}
\end{proof}

\subsection{Auxilliary Graphs and Trees: $H$, $\tilde{H}$, $T_0$, and $T_1$}

Refer to \figref{auxilliary}. Consider the auxilliary
graph $H$ with vertex set $V(H)\subseteq V(\dual{T})$ and whose edge set
consist of the edges of $C$ plus those edges of $\dual{T}$ that belong
to any face pinched by $C$. Let $v_0,\ldots,v_{r-1}$ be the cyclic sequence of vertices on some face $f$ of $\dual{T}$ that is pinched by $C$.  
We identify three kinds of vertices that are \emph{special} with respect to $f$:
\begin{enumerate}
  \item A vertex $v_i$ is special of \emph{Type~A} if $v_{i-1}v_i$ is an edge of $C$ and $v_iv_{i+1}$ is not an edge of $C$.
  \item A vertex $v_i$ is special of \emph{Type~B} if $v_{i-1}v_i$ is not an edge of $C$ and $v_iv_{i+1}$ is an edge of $C$.
  \item A vertex $v_i$ is special of \emph{Type~Y} if $v_i$ not incident to any edge of $C$ and $v_i$ has degree 3 in $H$.
\end{enumerate}

  \begin{figure}
     \begin{center}\begin{tabular}{cc}
		\includegraphics[width=.45\textwidth]{figs/t0t1-2} &
		\includegraphics[width=.45\textwidth]{figs/t0t1-3} \\
                (a) & (b) \\[1em]
		\includegraphics[width=.45\textwidth]{figs/t0t1-4} & 
		\includegraphics[width=.45\textwidth]{figs/t0t1-5} \\
                (c) & (d)
     \end{tabular}\end{center}
	  \caption{(a)~the cycle $C$ in $\dual{T}$ with faces classified as pinched or caressed; (b)~the auxilliary graph $H$; (c)~the auxilliary graph $\tilde{H}$ with keeper paths highlighted; (d)~the trees $T_0$ and $T_1$.}
	  \figlabel{auxilliary}
  \end{figure}


We say that a chord path $v_i,\ldots,v_j$ is a \emph{keeper} with respect to
$f$ if $v_i$ is special of Type~A, $v_j$ is special of Type~B, and none
of $v_{i+1},\ldots,v_{j-1}$ are special.  We let $\tilde{H}$ denote the
subgraph of $H$ containing all the edges of $C$ and all the edges of
all paths that are keepers with respect to some pinched face $f$ of $\dual{T}$.

It is worth emphasizing at this point that, by definition, every keeper is
entirely contained in the boundary of at least one face $f$ of $\dual{T}$.
This property will be useful shortly.

Let $\tilde{H}'$ denote the graph that is homeormophic to $\tilde{H}$
but does not contain any degree 2 vertices.  That is, $\tilde{H}'$
is the minor of $\tilde{H}$ obtained by repeatedly contracting an edge
incident a degree-2 vertex.  The graph $\tilde{H}'$ naturally inherits an
embedding from the embedding of $\tilde{H}$.  This embedding partitions
the edges of $\tilde{H}'$ into three sets:
\begin{enumerate}
  \item The set $B$ of edges that are contained in (the embedding of) $C$;
  \item The set $E_0$ of edges whose interiors are contained in the
  interior of (the embedding of) $C$; and
  \item The set $E_1$ of edges whose interiors are contained in the
  exterior of (the embedding of) $C$.
\end{enumerate}

Observe that, for each $i\in\{0,1\}$, the graph $H_i$ whose edges are
exactly those in $B\cup E_i$ is outerplanar, since all vertices of $H_i$
are on a single face, whose boundary is $C$.  Let $\dual{H_i}$ be dual
of $H_i$ and let $T_i$ be the subgraph of $\dual{H_i}$ whose edges are
all those dual to the edges of $E_i$. From the outerplanarity of $H_i$,
it follows that $T_i$ is a tree.  

Each vertex of $T_i$ corresponds to a face of $\tilde{H}$.  From this
point onwards, we will refer to the vertices of $T_i$ as \emph{nodes}
to highlight this fact, so that a node $u$ of $T_i$ is synonymous with
the subset of $\R^2$ contained in the corresponding face of $\tilde{H}$.
In the following, when we say that a node $u$ of $T_i$ contains a face $f$
of $\dual{T}$ we mean that $f$ is one of the faces of $\dual{T}$ whose
union makes up $u$.  The degree, $\delta_u$ of any node $u$ in $T_i$
is exactly equal to the number of keeper paths on the boundary of $u$.

The following lemma allows us to direct our effort towards proving that
one of $T_0$ or $T_1$ has many leaves.

\begin{lem}\lemlabel{one-caressed-leaf}
   Each leaf $u$ of $T_i$ contains at least one face of $\dual{T}$
   that is caressed by $C$.
\end{lem}

\begin{proof}
   The edge of $T_i$ incident to $u$ corresponds to a chord path $P$. The
   graph $P\cup C$ has two faces with $P$ on its boundary, one of which
   is $u$.  The lemma now follows immediately from \lemref{one-caressed},
   with $R=u$.
\end{proof}

We will make use of the following well-known property of
3-connected plane graphs.

\begin{lem}\lemlabel{one-shared-edge}
   If $T$ has $n\ge 4$ vertices then any two faces of $\dual{T}$ share at 
   most one edge.
\end{lem}

\begin{proof}
   Suppose that two faces $f$ and $g$ share two edges $e_1$ and
   $e_2$. Then $e_1$ and $e_2$ form an edge cutset of $\dual{T}$.
   If $\dual{T}$ contains at least four vertices, then two of the
   endpoints of $e_1$ and $e_2$ form a vertex cutset of $\dual{T}$
   of size 2, contradicting the fact that $\dual{T}$ is 3-connected.
   That $\dual{T}$ contains at least 4 vertices follows from Euler's
   Formula, which gives the number of vertices in $\dual{T}$ as $2n-4\ge
   4$ for all $n\ge 4$.
\end{proof}


\begin{lem}\lemlabel{many-caressed-or-high-degree}
   Let $u$ be a node of $T_i$ and let $\rho_u$, $\kappa_u$, and $\delta_u$ denote the number of pinched faces of $\dual{T}$ in $u$, the number of caressed faces of $\dual{T}$ in $u$, and the degree of $u$ in $T_i$, respectively.  Then $\rho_u \le 2(\kappa_u+\delta_u)$.
\end{lem}

Before proving \lemref{many-caressed-or-high-degree}, we point out that
the leading constant 2 is tight. \figref{tight} shows an example in
which all $\rho_u=2k+1$ pinched faces of $\dual{T}$ are contained in a
single (pink) node $u$ of $T_0$ that contains $\kappa_u=0$ caressed faces and
has degree $\delta_u=k+2$.

\begin{figure}
  \begin{center}
    \includegraphics{figs/tight}
  \end{center}
  \caption{An example showing the tightness of \lemref{many-caressed-or-high-degree}.}
  \figlabel{tight}
\end{figure}

\begin{proof}
   The proof is a discharging argument.  We assign each pinched face in 
   $u$ a single unit of charge, so that the total charge is $\rho_u$.
   We then describe a discharging procedure that preserve the total
   charge.  After executing this procedure, pinched faces in $u$
   have no charge, each caressed face in $u$ has charge at most 2,
   and each keeper path in $u$ has charge at most 2.  Since there is a
   bijection between keeper paths in $u$ and edges of $T_i$ incident to
   $u$, this proves the result.

   We now describe the discharging procedure, which are recursive and take as
   input a chord path $P$ that partitions $u$ into two parts $L$ and $R$.
   We require as a precondition that there are $m\ge 1$ pinched faces
   of $\dual{T}$ in $L$, each of which contains at least one edge of $P$ and
   such that every edge of $P$ is contained in at least one of these faces.
   During a recursive call, $P$ may have a charge $c\in\{0,1,2\}$. This
   charge will be at most 1 if $m\ge 1$, but can be 2 if $m=1$.

   To initialize the discharging procedure, we choose an arbitrary pinched
   face $f$ contained in $u$.  The face $f$ begins with one unit of charge
   and has $r\ge 2$ chord paths $P_1,\ldots,P_r$ on its boundary.
   We move the charge from $f$ onto $P_1$ and apply the recursive
   procedure to $P_1$, with a charge of 1 (with $L$ being the component
   of $u\setminus P_1$ that contains $f$).  We then recursively apply the
   discharging procedure on each of $P_2,\ldots,P_r$ with a charge of 0.

   Next we describe each recursive step, during which we are given $P$
   with some charge $c\in\{0,1,2\}$.  There are several cases to consider
   (see \figref{discharging}):
	\begin{figure}
		\begin{center}
		\begin{tabular}{cc}
			\includegraphics{figs/discharge-2} &
			\includegraphics{figs/discharge-3} \\
			 2.a & 2.b \\[1.5em]
			\includegraphics{figs/discharge-4} &
			\includegraphics{figs/discharge-5} \\
			 2.c & 3 
		\end{tabular}
		\end{center}
		\caption{Discharging steps in the proof of \lemref{many-caressed-or-high-degree}.}
		\figlabel{discharging}
	\end{figure}
  \begin{enumerate}
     \item $R$ contains no face of $\dual{T}$ that is pinched by $C$.
     This could occur if $R$ is empty, because $P$ is a keeper path,
     in which case we leave a charge of $c$ on it and we are done.
     Otherwise $R$ is non-empty and \lemref{one-caressed}  ensures that
     $R$ contains at least one caressed face $f$.  We move the charge
     from $P$ onto $f$ and we are done.

     \item $R$ contains a face $f$ that is pinched by $C$ and that shares
     at least one edge with $P$.  We consider three subcases:
     \begin{enumerate}
	\item $f$ contains neither endpoint of $P$. In this case,
	$R\setminus f$ has at least three connected components, $A$, $B$,
	and $X_1,\ldots,X_k$, where $A$ and $B$ each contain an endpoint
	of $P$ and each $X_i$ has a chord path $P_i$ in common
	with $f$.  We recurse on each of these components so that each
	of these components takes the place of $R$ in the recursion. When
	recursing on $A$ we take one unit of charge from $P$ (if needed)
	and place it on $A$'s chord path.  When recursing on $B$ we take
	the second unit of charge from $P$ (if needed) and place it on
	$B$'s chord path.  When recursing on $X_1$ we move the unit of
	charge from $f$ to $P_1$.  When recursing on $X_2,\ldots,X_k$
	we use no charge on $P_2,\ldots,P_k$.

	\item $f$ contains exactly one endpoint of $P$.  In this case,
	$u\setminus f$ has one connected component $A$ that contains
	an endpoint of $P$ and one or more connected components
	$X_1,\ldots,X_k$ where each $X_i$ has a chord path $P_i$ on the
	boundary of $f$.
	The path $P$ has a charge $c\le 2$.  When recursing on $X_1$
	we assign all of $P$'s charge to the chord path $P_1$, which is
	contained in the single pinched face $f$.
	When recursing on $A$ we move the single unit of charge from $f$
	to the chord path of $A$.

        \item $f$ contains both endpoints of $P$.  We claim that,
        in this case, $P$ must be on the boundary of more than one
        pinched faces in $L$, otherwise $P$ would be a keeper path.
        To see this, observe that the face $f$ contains both the first
        edge $e_1$ and last edge $e_2$ of $P$. If $e_1=e_2$ because $P$
        is a single-edge, then it is certainly a keeper, which is not
        possible. Otherwise, by
        \lemref{one-shared-edge}, $e_1$ and $e_2$ are on the boundary of
        two different faces in $L$.  By assumption, both of these faces
        are pinched by $C$.

        Therefore $P$ has at most one unit of charge assigned to it.
        Now, $R\setminus f$ has one or more connected components
        $X_1,\ldots,X_k$ sharing chord paths $P_1,\ldots,P_k$ with $f$
        on which we recurse.  When recursing on $X_1$ we move the charge
        from $P$ and the charge from $f$ to $P_1$.  When recursing on
        the remaining $X_i$, $i\in\{2,\ldots,k\}$ we assign no charge
        to $P_i$.
  \end{enumerate}
  \item $R$ contains at least one pinched face, but no pinched face in $R$
  shares an edge with $P$.  In this case, consider the face $g$ of $H$
  that is contained in $R$ and has $P$ on its boundary. By definition,
  $g$ contains no pinched faces of $\dual{T}$, but $g$ is touched by $C$,
  so $g$ contains at least one caressed face\footnote{In fact $g$ contains at
  least two good faces, one for each endpoint of $P$.} $f$ of $\dual{T}$.
  We move the $c$ units of charge from $P$ onto $f$.

  Now, $R$ still contains one or more pinched faces
  $f_1,\ldots,f_k$, where each $f_i$ shares a chord path $P_i$ with $g$.
  On each such face $f_i$, we run the initialization procedure described
  above except that we recurse only on the chord paths of $f_i$ that
  do not share edges with $g$. i.e., we do not recurse on the chord
  path $P_i$.
\end{enumerate}
   This completes the description of the discharging procedure, and the proof.
\end{proof}




%The following lemma says that if $C$ touches many faces then it must caress many faces, or $T_0$ and $T_1$ must have many degree-2 nodes.
%
%\begin{lem}\lemlabel{many-caressed-or-many-degree-2}
%  Let $\tau$, $\rho$, and $\kappa$ denote the number of nodes touched,
%  pinched, and caressed by $C$, respectively and let $m_2$ denote the
%  total degree of degree-2 nodes in $T_0$ and $T_1$.  Then $11\kappa +
%  4m_2 \ge \tau$.
%\end{lem}
%
%\begin{proof}
%  Since every face of $\dual{T}$ touched by $C$ is either pinched or
%  caressed, for any node $u$ of $T_0$ or $T_1$, we have
%  \begin{equation}
%       \kappa_u + \rho_u = \tau_u \enspace . \eqlabel{tpc}
%  \end{equation}
%  From \lemref{many-caressed-or-high-degree}, we have 
%  \[
%      2(\kappa_u+\delta_u) \ge \rho_u \enspace .
%  \]
%  Combining the preceding equations and summing over all $u$ in $T_0$
%  and $T_1$, we ge
%  \begin{equation}
%       3\kappa + 2\delta \ge \tau \enspace . \eqlabel{tpc-a}
%  \end{equation}
%  The total number of edges in $T_0$
%  and $T_1$ is $\delta/2$ and therefore the total number, $m$,
%  of nodes in $T_0$ and $T_1$ is $\delta/2+1$. Using the inequality
%  $2m\ge \delta$ in \eqref{tpc-a} we obtain 
%  \begin{equation}
%       3\kappa + 4m \ge \tau \enspace .  \eqlabel{tpc-b}
%  \end{equation}
%  The number, $m_{\ge 3}$ of nodes of degree at least three in $T_0$
%  and $T_1$ is less than the number, $m_1$, of leaves.  Therefore,
%  the number $m_2$ of degree 2 nodes is at least
%  \[
%     m_2 = m - m_1 - m_{\ge 3} \ge m-2m_1 \enspace .
%  \]
%  Using the inequality $m\le m_2+2m_1$  in \eqref{tpc-b}, we obtain:
%  \begin{equation}
%       3\kappa + 4(m_2+2m_1) \ge \tau \enspace .  \eqlabel{tpc-c}
%  \end{equation}
%  By \lemref{one-caressed}, each leaf contains at least one face caressed
%  by $C$, so $\kappa \ge m_1$. Using this inequality in \eqref{tpc-c}
%  we obtain
%  \begin{equation}
%       3\kappa + 4(m_2+2\kappa) \ge \tau \enspace .  \eqlabel{tpc-d}
%  \end{equation}
%  which simplifies to $11\kappa + 4m_2 \ge \tau$, as required.
%\end{proof}

\subsection{Bad Nodes}

%From \lemref{many-caressed-or-many-degree-2} we know that $C$ either
%caresses $\tau/22=\Omega(\tau)$ faces, or the total number of degree-2
%nodes in $T_0$ and $T_1$ is at least $\tau/8=\Omega(\tau)$.  This helps
%understand the example in \figref{few-caressed}, in which $C$ caresses
%only 4 faces and each of $T_0$ and $T_1$ is a path, so nearly all their
%nodes are degree-2. \note{PM}{\lemref{many-caressed-or-many-degree-2} is only ever used in this paragraph. Delete it?}

We say that a node of $T_i$ is \emph{bad} if it has degree 2 and
contains no face of $\dual{T}$ that is caressed by $C$.  We now move
from studying individual nodes of $T_0$ and $T_1$ to studying global
quantities associated with $T_0$ and $T_1$.  From this point on, for
each $i\in\{0,1\}$,
\begin{enumerate}
  \item $\tau_i$, $\rho_i$, and $\kappa_i$ refer the total numbers of faces contained in nodes of $T_i$ that are touched, pinched, and caressed by $C$, respectively; 
  \item $n_i$ refers to the number of nodes of $T_i$;
  \item $\delta_i = 2(n_i-1)$ is the total degree of all nodes in $T_i$; and
  \item $b_i$ is the number of bad nodes in $T_i$.
\end{enumerate}


\begin{lem}\lemlabel{few-caressed-implies-many-nodes}
%  Let $\tau_i$, $\rho_i$, and $\kappa_i$ denote the number of faces
%  in $\dual{T}$ contained in nodes of $T_i$ that are touched, pinched,
%  and caressed by $C$ and let $n_i$ denote the number of nodes of $T_i$.
  If $\kappa_i \le \tau_i/6$ then $n_i\ge \tau_i/8$.
\end{lem}

\begin{proof}
	From \lemref{many-caressed-or-high-degree} we know $\rho_i \le 2(\kappa_i+\delta_i)$, so
  \[
  \tau_i  = \kappa_i + \rho_i
     \le 3\kappa_i + 2\delta_i 
     = 3\kappa_i + 4(n_i-1) 
     \le \tau_i/2 + 4n_i \enspace ,
  \]
  and reorganizing the left- and right-hand sides gives the desired result.
\end{proof}

\begin{lem}\lemlabel{most-nodes-are-bad}
%   Let $n_i$ and $b_i$ denote the number of nodes and bad nodes in
%   $T_i$. 
   For any $0<\epsilon < 1$, if $b_i \le (1-\epsilon)n_i$, then
   $\kappa_i = \Omega(\epsilon\tau_i)$.
\end{lem}

\begin{proof}
   Partition the nodes of $T_i$ into the following sets:
   \begin{enumerate}
       \item the set $B$ of bad nodes;
       \item the set $N_1$ of leaves;
       \item the set $N_{\ge 3}$ of nodes having degree at least 3;
       \item the set $N_2$ of nodes having degree 2 that are not bad.
   \end{enumerate}
   \begin{align*}
     b_i & = n_i - |N_1|- |N_{\ge 3}|-|N_2| \\
	   & > n_i - 2|N_1|-|N_2| & \text{since $|N_1|\ge |N_{\ge 3}|$} \\
         & \ge  n_i - 2\kappa_i -|N_2| 
	   & \text{(since, by \lemref{one-caressed-leaf}, $\kappa_i\ge |N_1|$)} \\
           & \ge  n_i - 3\kappa_i 
           & \text{(since each node in $N_2$ contains a caressed face)}
    \end{align*}
    Thus, we have
    \[
          n_i-3\kappa_i \le b_i \le (1-\epsilon)n_i
    \]
    and rewriting gives 
    \begin{equation}
      \kappa_i \ge \epsilon n_i/3 \enspace . \eqlabel{blech}
    \end{equation}
    If $\kappa_i \ge \tau_i/6$, then the proof is complete.  On the other hand, if $\kappa_i \le \tau_i/6$ then, by \lemref{few-caressed-implies-many-nodes}, $n_i \ge \tau_i/8$. Combining this with \eqref{blech}  gives
    \[
      \kappa_i \ge \epsilon n_i/3 \ge \epsilon\tau_i/24 = \Omega(\epsilon \tau_i) \enspace . \qedhere
    \]
\end{proof}

\subsection{Interactions Between Bad Nodes}

We have now reached a point in which we know that the vast
majority of nodes in $T_0$ and $T_1$ are bad nodes, otherwise
\lemref{most-nodes-are-bad} implies that a constant fraction of the
faces touched by $C$ are caressed by $C$.  At this point, we are ready
to study interactions between bad nodes of $T_0$ and bad nodes of $T_1$.


\begin{lem}\lemlabel{common-face}
   If $u$ is a bad node then there is a single face $f$ of $\dual{T}$
   that is contained in $u$ and that contains all edges of $C\cap u$.
\end{lem}

\begin{proof}
   First observe that, since $u$ has degree 2, $C\cap u$ has exactly
   two connected components $C_1$ and and $C_2$. Thus $u$'s boundary
   consists of $C_1$, $C_2$ and two chord paths $P_1$ and $P_2$.

   If $C\cap u$ is not contained in a single face of $\dual{T}$, then
   some path $P$ with endpoints on the boundary of $u$ and internal
   vertices in the interior of $u$ separates two edges of $E(C)$ on the
   boundary of $u$.  There are a few cases to rule out:
   \begin{enumerate}
       \item $P$ has both endpoints on $C_i$ for some $i\in\{1,2\}$. In
       this case, $P$ is a chord path and, by \lemref{one-caressed}
       $u$ contains a face that is caressed by $C$, contradicting the
       assumption that $u$ is a bad node.

       \item $P$ has one endpoint on $C_i$ and one endpoint on $P_j$
       for some $i,j\in\{1,2\}$.  In this case, $P\cup P_j$ contains a
       chord path with both endpoints on $C_i$, again contradicting the
       assumption that $u$ is a bad node.

       \item $P$ has one endpoint on $P_1$ and one endpoint on $P_2$.
       In this case, $P\cup P_1\cup P_2$ contains a chord path with both
       endpoints on $C_1$, again contradicting the assumption that $u$
       is a bad node.

       \item $P$ has one endpoint on $C_1$ and one endpoint on $C_2$.
       The path $P$ is not a keeper, otherwise it would have split $u$
       into two nodes.	Therefore, it must be the case that $P$ contains
       an internal vertex.  Let $S_1$ be the set of internal vertices
       of $P$ and let $S_2$ be the set of vertices on the boundary
       of $u$, not including the endpoints of $P$.  Since $\dual{T}$
       is 3-connected, there is a path from $S_1$ to $S_2$ that does
       not contain any endpoint of $P$.  The shortest such path, $P'$,
       does not contain any edges of $P$.  Again, using portions of $P$,
       $P_1$, $P_2$, and $P'$ we can construct a chord path, contained
       in $u$, with both endpoints on $C_1$ or both endpoints on $C_2$,
       contradicting the assumption that $u$ is a bad node.
\end{enumerate}
\end{proof}

The following lemma shows that a bad node $u$ in $T_0$ and a bad node $w$
in $T_1$ share at most one edge of $C$.

\begin{lem}\lemlabel{bad-one-shared-edge}
   If $u$ and $w$ are bad nodes that share at least one edge of $C$, then
   $u$ and $w$ share exactly one edge of $C$.
\end{lem}

\begin{proof}
   Suppose $u$ and $w$ share two edges $e_1$ and $e_2$ of $C$.  Then,
   by \lemref{common-face}, there is a common face $f_u$ in $u$ that
   contains $e_1$ and $e_2$, Similarly, there is a common face $f_w$
   contained in $w$ that contains both $e_1$ and $e_2$.  But this
   contradicts \lemref{one-shared-edge}.
\end{proof}


\begin{lem}\lemlabel{one-edge-between}
   If $u$ and $w$ are bad nodes of $T_i$ sharing a common chord path $P$,
   then $P$ is a single edge.
\end{lem}

\begin{proof}
   By \lemref{common-face}, $u$ and $w$ have the first edge of $P$ in
   common and the last edge of $P$ in common.  \lemref{one-shared-edge}
   therefore implies that the first and last edge of $P$ are the same,
   so $P$ has only one edge.
\end{proof}

\subsection{Really Bad Nodes}


At this point we will start making use of the assumption that the
triangulation $T$ has maximum degree $\Delta$, which is equivalent to
the assumption that each face of $\dual{T}$ has at most $\Delta$ edges
on its boundary.

\begin{obs}\obslabel{degree-touched}
  If $T$ has maximum degree $\Delta$ and $C$ has length $\ell$, then
  the number of faces $\tau$ of $\dual{T}$ touched by $C$ is at least 
  $2\ell/\Delta$.  At least $\ell/\Delta$ of these faces are in
  the interior of $C$ and at least $\ell/\Delta$ of these faces are
  in the exterior of $C$.
\end{obs}

\begin{proof}
  Orient the edges of $C$ counterclockwise so that, for each edge
  $e$ of $C$, the face of $\dual{T}$ to the left of $e$ is in $C$'s
  interior and the face of $\dual{T}$ to the right of $e$ is in $C$'s
  exterior.  Each face of $\dual{T}$ has at most $\Delta$ edges.
  Therefore, the number of faces to the right of edges in $C$ is
  at least $\ell/\Delta$. The same is true for the number of faces
  of $\dual{T}$ to the left of edges in $C$.
\end{proof}

For a node $u$ of $T_i$, we define $N(u)$ as the set of nodes (in $T_0$
and $T_1$) that share an edge of $\dual{T}$ with $u$.  We say that a
node $u$ is \emph{really bad} if $u$ and all nodes in $N(u)$ are bad.

\begin{lem}\lemlabel{lots-of-really-bad}
  For every sufficiently small $0<\alpha < 1/2$,  
  if $T$ has maximum degree $\Delta$, $C$ has length $\ell$, and the number
  $\kappa$, of faces caressed by $C$ is at most $\alpha\ell/\Delta^2$, then
  the number of really bad nodes in $T_0$ is at least $n_0-O(\alpha n_0)$.
\end{lem}

\begin{proof}
  First consider the tree $T_0$.  From \obsref{degree-touched},
  we know that $\tau_0\ge \ell/\Delta$.  By assumption,
  $\kappa_0\le\kappa\le \alpha\ell/\Delta^2 < \tau_0/6$
  so, by \lemref{few-caressed-implies-many-nodes}, $n_0 \ge
  \tau_0/8=\Omega(\ell/\Delta)$.
  
  Let $\epsilon=a\alpha/\Delta$ for some $a>0$ to be discussed shortly.
  By \lemref{most-nodes-are-bad}, if $b_0 < (1-\epsilon) n_0$, then
  \[ 
      \kappa \ge \kappa_0 = \Omega(\epsilon\tau_0)
               = \Omega(\epsilon\ell/\Delta)
               = \Omega(a\alpha\ell/\Delta^2) 
               \enspace .
  \]
  For sufficiently large, constant, $a$ this violates our assumption
  that $\kappa \le \alpha\ell/\Delta^2$.
  Therefore, we may assume that $b_0\ge (1-\epsilon)n_0$, i.e., most nodes
  of $T_0$ are bad.

  We now want to study how many of the bad nodes in $T_0$ are really bad.
  Each node $w$ of $T_0$ is in $N(u)$ for at most $\delta_w$ bad nodes
  $u$ in $T_0$.  Let $A$ be the set of nodes in $T_0$ that are not
  bad and partition $A$ into $A_1$ (leaves), $A_2$ (degree-2 nodes)
  and $A_{\ge 3}$ (nodes of degree at least 3).  We make use of the
  following inequality:
	\begin{equation}
		|A_1| = 2 + \sum_{w\in A_3}(\delta_w-2) \ge \sum_{w\in A_3}(\delta_w-2)
		\ge \sum_{w\in A_3}3\delta_w \enspace ,  \eqlabel{triple}
	\end{equation}
  which is true because $x-2>x/3$ for all $x\ge 3$. 
  Now we have 
  \begin{align*}
   \sum_{w\in A}\delta_w 
    & = \sum_{w\in A_1}\delta_w 
         + \sum_{w\in A_2}\delta_w 
         + \sum_{w\in A_{\ge 3}}\delta_w \\
	  & \le |A_1| + 2|A_2| + 3|A_1| & \text{(by \eqref{triple})}\\
    & \le 6\kappa & \text{(since $\kappa_w\ge 1$ for each $w\in A_1\cup A_2$)}\\
    & \le 6\alpha\ell/\Delta^2 & \text{(by assumption)}\\
    & = O(\alpha n_0/\Delta) & \text{(since $n_0=\Omega(\ell/\Delta)$)}\enspace .
  \end{align*}
  That is, the set of non-bad nodes in $T_0$ prevents at most $O(\alpha
  n_0/\Delta)$ bad nodes of $T_0$ from being really bad. Next we account
  for how nodes in $T_1$ prevent bad nodes in $T_0$ from being really bad.

  Let $A'$ be the set of non-bad nodes in $T_1$.  For two
  nodes $u$ in $T_0$ and $w$ in $T_1$,  $w\in N(u)$ if and only if $w$
  and $u$ share an edge of $C$.  The number of edges of $C$ incident to a node $w$ is at most $(\Delta-1)\tau_w < \Delta\tau_w$.  Therefore, we can upper bound the number of bad nodes in $T_0$ that are prevented from being really bad by some node in $T_1$ as
  \begin{align*}
   \sum_{w\in A'}\Delta \tau_w  
    & \le  \sum_{w\in A'}2\Delta(\kappa_w + \delta_w) & \text{(by \lemref{many-caressed-or-high-degree})} \\
    & \le  2\Delta\kappa + \sum_{w\in A'}2\Delta\delta_w \\
    & < 2\Delta\kappa + 12\Delta\kappa & \text{(by the same argument as above)}\\
    & = 14\Delta\kappa \\
    & \le 14\alpha\ell/\Delta & \text{(by assumption)}\\
    & = O(\alpha n_0) & \text{(since $n_0=\Omega(\ell/\Delta)$)}
  \end{align*}
  Therefore, the number of bad nodes in $T_0$ is at least
  $(1-\epsilon)n_0$ and the number of these that are really bad is
  at least $(1-\epsilon)n_0 - O(\alpha n_0) = n_0-O(\alpha n_0)$,
  as required.
\end{proof}

For a node $u$ of $T_i$, we define $N^0(u) = \{u\}$ and, for any $i\in\N$,
we define $N^{i}(u)=\bigcup_{w\in N^{i-1}(u)} N(w)$.   We say that a node
$u$ in $T_i$ is \emph{really$^{i}$ bad} if $u$ is bad and all nodes in
$N^i(u)$ are bad.  The proof of the following lemma is a straightforward
generalization of the proof of \lemref{lots-of-really-bad}.

\begin{lem}\lemlabel{lots-of-really-i-bad}
  For any constant $i\in\N$ and every sufficiently small $0<\alpha < 1/2$,  
  if $T$ has maximum degree $\Delta$, $C$ has length $\ell$, and the number,
  $\kappa$, of faces caressed by $C$ is at most $\alpha\ell/\Delta^{i+1}$, then
  the number of really$^i$ bad nodes in $T_0$ is at least $n_0-O(\alpha n_0)$.
\end{lem}

For our purposes, it will be sufficient to work with bad ($i=0$), really
bad $(i=1$), and really really bad ($i=2$) nodes.

\subsection{Tree/Cycle Surgery}

We summarize the situation so far.  We are left with the case where
$C$ has length $\ell$ and therefore touches $\Omega(\ell/\Delta)$
faces.  To complete the proof of \thmref{main} we must deal with
the situation where $C$ caresses $o(\ell/\Delta^4)$ faces and
therefore has $o(\ell/\Delta^4)$ leaves (\lemref{one-caressed-leaf}),
$\Omega(\ell/\Delta)$ nodes (\lemref{few-caressed-implies-many-nodes}),
and the fraction of really really bad nodes in $T_0$ and $T_1$ is
$1-O(1/\Delta^2)$ (\lemref{lots-of-really-i-bad}). 


%\figref{few-caressed}
%illustrates an extreme example of this, in which $C$ caresses only 4
%faces of $\dual{T}$, and $T_0$ and $T_1$ each have only 2 non-bad nodes,
%X non-really bad nodes, and Y non-really$^2$ bad nodes.  

To handle cases like these, the only option is to perform surgery on
the cycle $C$ to increase the number of caressed nodes. In particular,
our strategy is to perform modifications to $C$ that increase the number
of faces caressed by $C$.
%Before finishing up, we need a final
%lemma that describes which nodes of $T_0$ and $T_1$ survive when the
%cycle $C$ is modified:
%
%\begin{lem}\lemlabel{survivors}
%   Let $C$ and $C'$ be two cycles in $\dual{T}$ and, for each $i\in\{0,1\}$, let $T_i$ and $T_i'$ be the resulting trees.  Then a node $u$ in $T_i$ appears in $T_i'$ if the following conditions are satisfied:
%  \begin{compactenum}
%     \item $u\cap C = u\cap C'$; and
%     \item for every node $w$ adjacent to $u$ in $T_i$ and every face $f$ of $\dual{T}$ in $w$, $f$ is pinched by $C$ if and only if $f$ is pinched by $C'$.
% \end{compactenum}
%\end{lem}
%
%\begin{proof}
%   This proof is just definition checking.
%   The node $u$ appears in $T_i'$ if (i)~each of the keeper paths
%   (for $C$) on the boundary of $u$ are also keeper paths for $C'$ and
%   (ii)~there is no keeper path for $C'$ that contains edges in the
%   interior of $u$.  The latter condition is satisified, since a keeper
%   path is always contained in a single pinched face and, since $u\cap
%   C=u\cap C'$, the set of faces of $\dual{T}$ contained in $u$ that are
%   pinched by $C$ is exactly the same as the set of faces contained in
%   $u$ that are pinched by $C'$.
%
%   Thus, all we need to verify is that the keeper paths for $C$ on the
%   boundary of $u$ are also keeper paths for $C'$.  Every such keeper path
%   $P$ is on the boundary of $u$ and some neighbour $w$ of $u$ in $T_i$.
%   Conditions 1 and 2 of the lemma ensure that every face of $\dual{T}$
%   that shares an edge with $P$ is pinched by $C$ if and only if it is
%   pinched by $C'$.  Therefore $P$ is a keeper path in $C$ if and only
%   if it is a keeper path in $C'$.
%\end{proof}
At this point we are ready to complete the proof of \thmref{main}.

\begin{proof}[Proof of \thmref{main}]
By \lemref{cycle-to-curve}, it suffices to prove the existence of
a cycle $C$ in $\dual{T}$ that caresses $\Omega(\ell/\Delta^4)$
faces.  We begin by applying \lemref{lots-of-really-i-bad} with
$i=2$ and $\alpha = \epsilon/\Delta$.  For sufficiently small, but constant,
$\epsilon$, \lemref{lots-of-really-i-bad} implies that $\kappa =
\Omega(\ell/\Delta^4)$ or the number of nodes in $T_0$ that are not really
really bad is at most $O(\epsilon n_0/\Delta)$.  In the former case,
$C$ caresses $\Omega(\ell/\Delta^4)$ faces of $\dual{T}$ and we are done.

In the latter case, consider the forest obtained removing
all nodes of $T_0$ that are not really really bad.  This forest has
$(1-O(\epsilon/\Delta))n_0$ nodes and $O(\epsilon n_0/\Delta)$ components,
each of which is a path.  At least one of these components contains
$\Omega(\Delta/\epsilon)$ nodes. In particular, for a sufficiently
small constant $\epsilon$, one of these components, $X$, has at least
$5\Delta$ nodes.

%The nodes in the path $X$ have a particularly simple structure. Each
%node $u$ in $X$ is bad and is adjacent, in $T_0$ to two bad nodes so,
%by \lemref{one-edge-between}, $u$ has exactly two edges $e_1$ and $e_2$
%that are not in $C$.  Furthermore, by \lemref{common-face}, all four
%endpoints of $e_1$, and hence $e_1$ and $e_2$ belong to a common face
%of $\dual{T}$ that also contains all edges of $u\cap C$.  Therefore,
%each node $u$ in $X$ consists of a single face of $\dual{T}$.

Consider some node $u$ in $X$, and let $C_a$ and $C_b$ be the
two components of $u\cap C$. Observe that $T_1[N(u)]$ consists
of two paths $a_1,\ldots,a_r$ and $b_1,\ldots,b_s$ of bad nodes
where each $a_1,\ldots,a_r$ contains an edge of $C_a$ and each of
$b_1,\ldots,b_r$ contains an edge of $C_b$. Note that it is
possible that $a_i=b_j$ for some values of $i$ and $j$, but everything
stated thus far, and subsequently, is still true. It follows from
\lemref{bad-one-shared-edge} that among any sequence of $\Delta$
consecutive nodes in $X$, at least one node has $|N(u)|\ge 5$ because
$r\ge 2$.  Let $u$ be any such node that is not among the first $2\Delta$
or last $2\Delta$ nodes of $X$.  Such a $u$ always exists because $X$
contains at least $5\Delta$ nodes.

Let $x_0=u$. We now describe some of the nodes in the vicinity of $u$ (refer to \figref{rrbad}):
\begin{enumerate}
  \item there is a path
    $x_{2\Delta},\ldots,x_1,x_0,y_1,\ldots,y_{2\Delta}$
    in $T_0$ consisting entirely of really really bad nodes.
  \item some really bad node $a_1$ of $T_1$ shares an edge with each of
    $x_0,\ldots,x_i$ for some $i\in\{1,\ldots,\Delta-4\}$.
  \item some really bad node $a_2$ of $T_1$ shares an edge with $a_1$ and
    and edge with $x_0$. 
  \item some really bad node $a_0$ of $T_1$ shares an edge with $a_1$ and
    with each of $x_i,\ldots,x_{i+j}$ for some $j\in\{0,\ldots,\Delta-4\}$.
\end{enumerate}

\begin{figure}
   \begin{center}\includegraphics{figs/second-try-1}\end{center}
   \caption{Nodes in the vicinity of $u=x_0$.}
   \figlabel{rrbad}
\end{figure}

The surgery we perform focuses on the nodes $u$ and $a_1$.  Consider
the two components of $C\cap a_1$. One of these components, $p$, shares
an edge with $u$.  By \lemref{bad-one-shared-edge}, the other component,
$q$, does not share an edge with $u$.  Imagine removing $u$ from $T_0$,
thereby separating $T_0$ into a component $T_x$ containing $x_1$ and a
component $T_y$ containing $y_1$.  Equivalently, one can think of removing
the edges of $u$ from $C$ separating $C$ into two paths $C_x$ and $C_y$
on the boundary of $T_x$ and $T_y$, respectively.  We distinguish between
two major cases (see \figref{main-cases}):

\begin{figure}
   \begin{center}
     \begin{tabular}{ccc}
       \includegraphics{figs/cases-2} &
       \includegraphics{figs/cases-3} &
       \includegraphics{figs/cases-4} \\
       (1) & (2) & $C'$
     \end{tabular}
   \end{center}
   \caption{Cases 1 and 2 in the proof of \thmref{main} and the surgery performed in Case~2.}
   \figlabel{main-cases}
\end{figure}
\begin{enumerate}
   \item $q\subset C_x$.  In this case, we punt to Case~2. By
   \lemref{one-edge-between} $a_1-C$ consists of two edges and exactly
   one of these edges, $e$, is not incident to $u$. Instead, $e$ is
   incident to    $x_i$.  We set $u'=x_i$, $x_1'=x_{i-1}$, $y_1'=x_{i+1}$,
   and $a_1'=a_1$. Observe that $a_1'$ connects the two components of
   $T_0-u'$ and shares edges with $u'$ and $x_1'$. This is exactly the
   situation considered in Case~2.

   \item $q\subset C_y$.  At this point it is helpful to think of
   $T_0$, $T_1$, and $C$ as a partition of $\R^2$, where nodes of $T_0$
   are coloured red, nodes of $T_1$ are coloured blue and $C$ is the
   (purple) boundary between red and blue.  To describe our modifications
   of $C$, we imagine changing the colours of nodes.  The effect that
   such a recolouring has on $C$ is immediately obvious: It produces a
   1-dimensional set $C'$ that contains every (purple) edge contained in
   the red-blue boundary. The set $C'$ is a collection of vertices and
   edges of $\dual{T}$. Therefore, if $C'$ is a simple cycle, then $C'$
   defines a new pair of trees $T_0'$ and $T_1'$.

%   The relationship between $(T_0,T_1)$ and $(T_0',T_1')$
%   takes more effort to understand.  This difficulty comes from the fact
%   that there is no bijection between the nodes in $V(T_0)\cup V(T_1)$
%   and the nodes in $V(T_0')\cup V(T_1')$.

   Refer to right two thirds of \figref{main-cases} for a simple
   (and misleading) example of what follows. For a full example,
   refer to \figref{recolouring}.  The surgery we perform, recolours
   $x_0,x_1,\ldots,x_{i-1}$ blue and recolours $a_1$ red.  Observe that,
   because $q\subset C_y$ and $p$ contain an edge of $x_1$, this implies
   that the red subset of $\R^2$ is simply-connected and its boundary $C'$
   is a simple cycle consisting of edges of $T^*$.  The new trees $T_0'$
   and $T_1'$ are therefore well defined.  We now make two claims that
   will complete our proof.

   \begin{figure}
     \begin{center}
       \begin{tabular}{cc}
         \includegraphics{figs/surgery-1} &
         \includegraphics{figs/surgery-2} 
       \end{tabular}
     \end{center}
     \caption{Performing surgery on $C$ to obtain $C'$ that caresses $a_0$.}
     \figlabel{recolouring}
   \end{figure}

   \begin{clm}\clmlabel{only-bad}
      For each $i\in\{0,1\}$, and every node of $w$ $T_i$ that is not bad,
      $C\cap w=C'\cap w$.  (Equivalently, for every face $f$ of $\dual{T}$
      that is not a bad node of $T_0$ or $T_1$, $C\cap f=C'\cap f$.)
   \end{clm}

   \begin{clm}\clmlabel{a0-caressed}
      The face $a_0$ is caressed by $C'$.
   \end{clm}

   These two claims complete the proof because, together,
   they imply that $C'$ caresses one more node than $C$.  Indeed, by
   definition, $C$ did not caress any faces belonging to bad nodes.
   Therefore, the first claim implies that the faces of $\dual{T}$
   caressed by $C'$ are a superset of those caressed by $C$.  The face
   $a_0$ is a bad node of $T_i$ so it is not caressed by $C$ but the
   second claim states that it is caressed by $C'$.  Therefore $C'$
   caresses at least one more face than $C$.

   This surgery recolours at most $\Delta-2\le \Delta$ nodes of $T_0$
   and $T_1$, so the difference in length between $C$ and $C'$
   is at most $\Delta^2$.  If we start with a cycle $C$
   of length $\ell$, then we can perform this surgery at least
   $\ell/(4\Delta^2)$ times before the length of $C$ decreases
   to less than $\ell'=\ell/2$. If at some point during this process,
   we are no longer able to perform this operation, it is because
   $C$ caresses $\Omega(\ell'/\Delta^4)=\Omega(\ell/\Delta^4)$
   faces of $\dual{T}$ and we are done.  By the end of this
   process, the number of faces caressed by $C$ is at least
   $\ell/(4\Delta^2)\in\Omega(\ell/\Delta^2)\subset\Omega(\ell/\Delta^4)$
   and we are also done.

   Thus, all that remains is to prove \clmref{only-bad} and
   \clmref{a0-caressed}.

   To prove \clmref{only-bad} we observe that $C$ and $C'$ differ only on
   the boundaries of nodes that are recoloured.  Thus, it is sufficient
   to show that all nodes in $R=\cup\{N(v): v\in\{x_0,\ldots,x_{i-1},a_1\}$
   are bad.  But this is immediate since $x_0,\ldots,x_{i-1}$ are really
   really bad and $a_1\in N(x_0)$, so $a_1$ is bad.  Since every node in $R$
   share an edge with at least one of $\{x_0,\ldots,x_{i-1},a_1\}$, every
   node in $R$ is therefore bad, as required.

   To prove \clmref{a0-caressed} we consider the boundary of the face
   $a_0$ of $T^*$ after the recolouring operation.  This boundary
   consists of, in cyclic order:
   \begin{enumerate}
     \item  An edge shared between $a_0$ and $a_1$.  This edge is in
       $C'$ since $a_0$ is in $T_1'$ and $a_1$ is in $T_0'$.

     \item A path of edges shared with $x_i,\ldots,x_{i+j}$.
      The nodes $x_i,\ldots,x_{i+j}$ are in $T_0$ and are distinct from
      $x_0,\ldots,x_{i-1}$, so these nodes are in $T_0'$.
      Therefore, this part of the bounary of $a_0$ is contained in $C'$.

     \item An edge shared between $a_0$ and another node $a_{-1}=\neq
     a_1$ of $T_1$. The faces of $a_{-1}$ are in $T_1'$ because $a_1$
     is the only face that moves from $T_1$ to $T_0'$. ($T_1'$ is the
     only face whose colour goes from blue to red.)

     \item A path of edges that contains at least one edge of $C_y$. If
     we fix an embedding in which the outer face is some face of $T_1$
     other than $a_0$, then this path contains a portion of $C$ that
     is traversed in clockwise order. By \lemref{bad-one-shared-edge}
     This path does not contain any edge of $x_i$.  Furthermore, this
     path does not contain any edges of $x_0,x_1,\ldots,x_{i-1}$ that
     are not on the outer face of $T_0\cup a_1$.  Therefore, this path
     consists of a (possibly empty) sequence of edges that are shared with
     $x_0,\ldots,x_{i-1}$ followed by a sequence of edges from $C_y$. The
     former part of this path is shared with nodes in $T_1'$, so these
     edges are not in $C'$.  The latter part of this path is shared with
     nodes in $T_y$, which are all contained contained in $T_0'$.
  \end{enumerate} 
  Therefore the intersection $C'\cap a_0$ consists of one connected
  component so $a_0$ is caressed by $C'$. \qedhere
\end{enumerate}
%We also
%name some other nodes of $T_1$: (i)~$a_0$ adjacent to $a_1$; (ii)~$a_{r+1}$
%adjacent to $a_r$; (iii)~$b_0$ adjacent to $b_1$; (iv)~$b_{s+1}$ adjacent
%  to $b_s$; (v)~the path $x_1,\ldots,x_{2\Delta}$ in $T_0$ that does not contain $u$; and (vi)~the
%  path $y_1,\ldots,y_{2\Delta}$ in $T_0$ that does not contain $u$.
%  
%Observe that the path $x_{2\Delta},\ldots,x_1,u,y_1,\ldots,y_{2\Delta}$
%consists entirely of really really bad nodes. The four nodes $a_0$, $a_{r+1}$,
%$b_0$, and $b_{s+1}$ are really bad because they are adjacent to at least
%one of these really really bad nodes.
%
%At this point it is helpful to think of $T_0$, $T_1$, and $C$ as a
%partition of $\R^2$, where nodes of $T_0$ are coloured red, nodes of $T_1$
%are coloured blue and $C$ is the (purple) boundary between red and blue.
%To describe our modifications of $C$, we imagine changing the colours
%of nodes.  The effect such a recolouring has on $C$ is immediately
%obvious: The new cycle $C'$ contains every (purple) edge contained in
%the red-blue boundary. This new cycle $C'$ defines a new pair of trees
%$T_0'$ and $T_1'$.  While the relationship between $C$ and $C'$ is easy
%to understand, the relationship between $(T_0,T_1)$ and $(T_0',T_1')$
%takes more effort to understand.  This difficulty comes from the fact
%that there is no bijection between the nodes in $V(T_0)\cup V(T_1)$
%and the nodes in $V(T_0')\cup V(T_1')$.
%
%There are several cases to consider:
%
%\begin{enumerate}
%   \item The path in $T_1$ from $a_1$ to $b_1$ does not contain $a_2$ or $b_2$.
%     Refer to \figref{case1-surgery}.
%\begin{figure}
%   \begin{center}
%     \includegraphics{figs/second-try-2}
%     \includegraphics{figs/second-try-3}
%   \end{center}
%   \caption{Case 1 of the cycle surgery.}
%   \figlabel{case1-surgery}
%\end{figure}
%
%     The purple boundary $a_1\cap C$ has two connected components: $C_1$
%     that contains an edge of $u$ and $C_2$.  Let $z_1,\ldots,z_t$ be
%     the nodes of $T_0$ in $N(a_1)$ that contain an edge of $C_2$. Note
%     that $z_1,\ldots,z_t$ form a path of bad nodes in $T_0$, since $u$
%     is really really bad, $a_1\in N(u)$ and $z_1,\ldots,z_t\in N(a_1)$. We
%     order $z_1,\ldots,z_t$ so that $z_1\in N(a_0)$.  Finally, let
%     $z_0,z_{t+1}\not\in\{z_1,\ldots,z_t\}$ be the two nodes of $T_0$
%     adjacent to $z_1$ and $z_t$, respectively. 
%     %\note{PM}{This is not
%     %completely unambiguous in the case $t=1$, but it doesn't matter for
%     %what comes next.}  
%     Note that $z_0$ and $z_{t+1}$ are not necessarily
%     bad nodes.
%
%     The forest $T_0-\{u\}$ has two components, one of which contains $x_1$
%     and the other contains $y_1$.  This latter component also contains
%     $z_0,\ldots,z_{t+1}$.  The forest $T_1-\{a_1\}$ has two components
%     one of which contains $a_r$ and the other of which contains $b_1$.
%
%     We make the following \emph{surgical modification} to $C$: we move
%     $a_1$ to $T_0$ and we move $u$ to $T_1$.  Moving $u$ to $T_1$
%     restores a path in $T_1$ from $a_r$ to $b_1$.  Moving $a_1$ to
%     $T_0$ restores a path in $T_0$ from $y$ to $x$ (through $z_t$).
%     We call the new cycle that results from this operation $C'$ and
%     the resulting trees $T_0'$ and $T_1'$.
%     
%     \begin{clm}
%       Every node of $T_0$ and $T_1$ is also present in $T_0'$ and $T_1'$,
%       respectively, except possibly for the following bad nodes: $u$,
%       $x_1,\ldots,x_{\Delta-1}$, $a_1,\ldots,a_{r}$, $b_1,\ldots,b_{s}$,
%       and $z_1,\ldots,z_t$.
%     \end{clm}
%
%     \begin{proof}[Proof of Claim]
%     For most nodes $v$ in $T_0$ or $T_1$, this claim follows immediately
%     from \lemref{survivors} since, for most nodes $v\cap C=v\cap C'$
%     and $w\cap C=w\cap C'$ for every $w\in N(v)$.  The only nodes requiring 
%     special consideration are the following:
%     \begin{enumerate}
%       \item[$z_0$:]  For this node $z_0\cap C=z_0\cap C'$ and $w\cap
%       C=w\cap C'$ for all $w\in N(z_0)\setminus z_1$.  Thus, the only
%       question is whether the chord path $P$ separating $z_0$ and $z_1$
%       is a keeper for $C'$.
%
%       We claim that $z_1$ contains a single face of $\dual{T}$.  By
%       \lemref{common-face}, there is a single face $f$ of $\dual{T}$ in
%       $z_1$ that contains $z_1\cap C$ and, by \lemref{one-edge-between},
%       this face also contains the single edge of $\dual{T}$ separating
%       $z_1$ and $z_2$.  Suppose now, for the sake of contradiction,
%       that $f$ does not contain the chord path $P$ separating $z_0$
%       and $z_1$.  The face $f$ must, however, contain the first and last
%       edge of $P$.  All edges of $f$ are in the auxilliary graph $H$ and
%       therefore $P$ contains an internal node that has degree 3 in $H$.
%       But this contradicts the fact that $P$ is a keeper for $C$.
%
%       Therefore $f$ contains $P$ on its boundary. Furthermore, $f$ is
%       pinched by $C'$, so $P$ is in the auxilliary graph $H'$ for $C'$.
%       Since the faces of $z_0$ pinched by $C$ and $C'$ are the same and
%       $P$ is a keeper for $C$, it follows that $P$ is a keeper for $C'$.
%
%     \item[$z_{t+1}$:] The node $z_{t+1}$ appears in $T_0'$ for the same
%     reason $z_0$ appears in $T_0'$.
%
%      \item[$y_1$:] The path $u,y_1,y_2,y_3$ consists entire of bad
%      nodes. Therefore, $y_1$ and $y_2$ are
%      faces of $\dual{T}$. The node $y_2$ is pinched by $C'$, therefore
%      the edge $e$ common to $y_1$ and $y_2$ is a keeper. Therefore all
%      edges of $y_1$ are in $C'\cup e$, so $y_1$ is a node of $T_0'$.
%         
%        
%    \item[$x_\Delta$:] That $x_\Delta$ is a node of $T_0'$ follows from
%    the fact that $x_{\Delta-1},x_{\Delta},x_{\Delta+1}$ is a path of
%         bad nodes in $T_0$ and $x_{\Delta}\cap C=x_{\Delta}\cap C'$.
%
%       \item[$a_0$:] The node $a_0$ appears in $T_1'$ as a leaf for exactly the same reason $y_1$ appears in $T_0'$ as a leaf.
%
%       \item[$b_0,b_{s+1},a_{r+1}$:] The nodes $b_0$, $b_{s+1}$, and $b_{r+1}$ appear in $T_1'$ for the same reasons
%       that $x_\Delta$ appears in $T_0'$.
%     \end{enumerate}
%     This completes the proof of the claim.
%     \end{proof}
%     The key point to take from all of this is that every node of $T_0$
%     and $T_1$ that is not bad is also present in $T_0'$ and $T_1'$.
%     %Furthermore, every non-bad node $w$ in $T_i$ has the same degree in $T_i$ and in $T_i'$.  
%     Furthermore, the nodes $y_0$ and $a_0$, which were bad nodes in
%     $T_0$ and $T_1$ are now leaves of $T_0'$ and $T_1'$.  Therefore,
%     for each $i\in\{0,1\}$, the number of leaves in $T_i'$ is greater
%     than the number of leaves in $T_i$.
%
%   \item The path in $T_1$ from $a_1$ to $b_1$ contains $b_2$ but not $a_2$.
%     Refer to \figref{case2-surgery}.
%   The surgery, and the proof, for this case is \emph{verbatim}
%   the same as Case~1. (Indeed, reread Case~1 while referring to
%   \figref{case2-surgery}.
%
%\begin{figure}
%   \begin{center}
%     \includegraphics{figs/second-try-x-2}
%     \includegraphics{figs/second-try-x-3}
%   \end{center}
%   \caption{Case 2 of the cycle surgery.}
%   \figlabel{case2-surgery}
%\end{figure}
%
%
% \item The path in $T_1$ from $a_1$ to $b_1$ contains $a_2$ and $b_2$.  This case is symmetric to Case~1.
%
%  \item The path in $T_1$ from $a_1$ to $b_1$ contains $a_2$ but not $b_2$.  This case is symmetric to Case~2.
%\end{enumerate}
%
%In all four cases, the surgery on $C$ increases the number of leaves
%in each of $T_1$ and $T_2$.  Furthermore, the only edges of $C$ that
%are removed by this surgery are on the boundary of $u$.  Therefore,
%this operation decreases the length of $C$ by at most $\Delta$.
%
%If we start with a cycle $C$ of length $\ell$, then we can perform
%this operation at least $\ell/(2\Delta)$ times before the length of
%$C$ decreases to less than $\ell'=\ell/2$. If at some point during
%this process, we are no longer able to perform this operation, it is
%because $C$ caresses $\Omega(\ell'/\Delta^4)=\Omega(\ell/\Delta^4)$
%faces of $\dual{T}$ and we are done.  By the end of this process,
%the number of leaves in each of $T_0$ and $T_1$ is at least
%$\ell/(2\Delta)\in\Omega(\ell/\Delta)\subset\Omega(\ell/\Delta^4)$ and we
%are also done.  This completes the proof of \thmref{main}.
\end{proof}

%\note{PM}{There's an annoying lopsidedness here. We can do surgeries $\Omega(\ell/\Delta)$ times before the length of $C$ really decreases. Unfortunately, we need the number of caressed faces to be $O(\ell/\Delta^4)$ in order to guarantee that we can perform a surgery. There's gotta be a better way\ldots}


\section{Discussion}
\seclabel{discussion}

%\note{PM}{Add a discussion here about the implications.}

It remains an open problem to eliminate the dependence of our results
on the maximum degree, $\Delta$, of $T$.  The next significant step
is to resolve the following conjecture:

\begin{conj}
  If $T$ is a triangulation whose dual $\dual{T}$ has a cycle of length
  $\ell$, then $\dual{T}$ has a cycle that caresses $\Omega(\ell)$ faces.
\end{conj}



\bibliographystyle{plain}
\bibliography{collsets}

\end{document}









