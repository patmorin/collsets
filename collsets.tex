\documentclass{patmorin}
%\usepackage[top=0.85in, bottom=0.85in, left=0.85in, right=0.85in]{geometry}

%%%%%%%%%%%%%%%%%%%%%%%%%%%%%%%%%%%%%%%%%%%%%%%%%%%%%%%%%%%%%%%%%%%%%%%%%%%%%%%%%%%%%%%%%%
%% Packages
%%%%%%%%%%%%%%%%%%%%%%%%%%%%%%%%%%%%%%%%%%%%%%%%%%%%%%%%%%%%%%%%%%%%%%%%%%%%%%%%%%%%%%%%%%
\usepackage{amsmath}
\usepackage{amsfonts}
\usepackage{amsthm}
\usepackage{graphicx}
%\usepackage{xspace}
%\usepackage{wrapfig}
\newenvironment{wrapfigure}[9]{\begin{figure}}{\end{figure}}
\newcommand{\Vspace}[1]{}
\usepackage{enumerate}
\usepackage{cite}
\usepackage{pat}
\usepackage{paralist}
\usepackage{hyperref}
\hypersetup{colorlinks=true, linkcolor=linkblue,  anchorcolor=linkblue,
citecolor=linkblue, filecolor=linkblue, menucolor=linkblue,
urlcolor=linkblue, pdfcreator=Me, pdfproducer=Me} 
\setlength{\parskip}{1ex}
\usepackage{algorithm}
\usepackage{subfig}
%\usepackage{subfigure}
\usepackage{array}
\usepackage[noend]{algpseudocode}
\usepackage[usenames]{xcolor}
\usepackage{stmaryrd} % for lightning-symbol
\usepackage{mathtools} % for mathclap in \conf-definition
\usepackage{todonotes}
%\usepackage{compress}
%\usepackage{times}
\usepackage{lineno}

\title{\MakeUppercase{Dual Cycles and Collinear Sets in Triangulations}%
    \thanks{This work was partly funded by NSERC and MRI.}}

\author{Vida Dujmovi\'c\thanks{Department of Computer Science and Electrical Engineering, University of Ottawa}\,\, and 
        Pat Morin\thanks{School of Computer Science, Carleton University}}


\begin{document}
\maketitle


\begin{abstract}
   We show that, if a triangulation $T$ of maximum degree $\Delta$
   has a dual that contains a cycle of length $\ell$, then $T$ has
   a plane straight-line drawing in which some \emph{collinear set}
   of $\Omega(\ell/\Delta^2)$ vertices lie on a line.  Using the
   current lower bounds on the length of longest cycles in 3-regular
   3-connected graphs, this implies that $T$ contains a collinear set
   of size $\Omega(n^{0.8}/\Delta^2)$.  Such collinear sets have numerous
   results in graph drawing.
\end{abstract}

\section{Introduction}

For a planar graph $G$, we say that a set $S\subseteq V(G)$ of vertices is
a \emph{collinear set} if $G$ has a non-crossing straight-line drawing
in which the vertices of $S$ are all collinear. The \emph{circumference},
 $c(G)$ of a graph $G$ is the length of its longest cycle. We prove the following
theorem:

\begin{thm}\thmlabel{main}
  Let $T$ be a triangulation of maximum degree $\Delta$ whose dual graph
  has circumference $\ell$. Then $T$ has a collinear set of
  size $\Omega(\ell/\Delta^2)$.
\end{thm}

The dual of a triangulation is a 3-connected cubic planar graph.
The study of the circumference of 3-connected cubic planar graphs has
a long history beginning with Tait \cite{X}, who conjectured that every
such graph is Hamiltonian.  This conjecture was disproved by Tutte who
gave a non-Hamiltonian 46-vertex example. Repeatedly replacing vertices
of this graph with copies of itself gives a family of graphs in which
$n$-vertex members have circumference $O(n^a)$, for $a=\log_{44}(45)) <
n^{0.993}$. The current best upper bound of this type is due to Gr\"unbaum
and Walther \cite{grunbaum.walther:shortness} who construct a 24-vertex
non-Hamiltonian cubic 3-connected planar graph, resulting in a family
of graphs in which $n$-vertex members have circumference $O(n^{\alpha})$
for $\alpha=\log_{23}(22)< 0.985$.

A series of results has steadily improved the lower bounds on the length
of the longest cycle in (not necessarily planar) 3-connected cubic
graphs \cite{X,X,X}.  The current record is held by Liu, Yu, and Zhang
\cite{liu.yu.zhang:circumference} who show that $n$-vertex 3-connected
cubic graph has a cycle of length $\Omega(n^{0.8})$.
Together with \thmref{main}, this result implies the following corollary:

\begin{cor}\corlabel{main}
  Every $n$-vertex triangulation of maximum degree $\Delta$ contains a
  collinear set of size $\Omega(n^{0.8}/\Delta^2)$.
\end{cor}


A longstanding open problem on circuference is \emph{Barnette's
Conjecture}, which asserts that every \emph{bipartite} 3-connected cubic
planar graph is Hamiltonian. Note that, if Barnette's conjecture is
true, this would imply that every triangulation in which every vertex
has even degree contains a collinear set of linear size.  
\comment{TODO: Think
about whether this even makes sense (for example, the assumption can
only be true when $n$ is odd).}



\section{Proof of \thmref{main}}

Let $G$ be a plane graph. A \emph{proper good curve} $C$ for $G$ is a
Jordan curve with the following properties:
\begin{enumerate}
  \item For any edge $xy$ of $G$, $C$ either contains $xy$, intersects
  $xy$ in a single point (possibly an endpoint) or $C$ is disjoint
  from $xy$.  
  \item $C$ contains at least one point in the interior of
  the outer face of $T$.
\end{enumerate}

Da Lozzo \etal\ show that proper good curves define collinear sets:

\begin{thm}
  In a plane graph $G$, a set $S\subseteq V(G)$ is a collinear set if
  and only if there is a proper curve for $G$ that contains $S$.
\end{thm}

To prove \thmref{main}, we will demonstrate a relationship between proper good curves in triangulations and cycles in their duals.  


\section{}

\newpage
\bibliographystyle{plain}
\bibliography{freecoll}

\end{document}









